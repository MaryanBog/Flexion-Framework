\documentclass[12pt]{article}

\usepackage[utf8]{inputenc}
\usepackage[T1]{fontenc}
\usepackage{lmodern}
\usepackage{amsmath, amssymb}
\usepackage{geometry}
\usepackage{hyperref}
\usepackage{setspace}
\usepackage{enumitem}

\geometry{a4paper, margin=1in}
\onehalfspacing

\title{Flexion Collapse (FC) V1.0 \\[4pt]
\large Structural Termination and Collapse Geometry in the Flexion Framework}
\author{Maryan Bogdanov}
\date{2025}

\begin{document}

\maketitle

\begin{abstract}
    Flexion Collapse (FC) V1.0 formalizes the terminal phase of structural evolution within the 
    Flexion Framework. Collapse is defined as the irreversible approach of the state vector 
    $X = (\Delta, \Phi, M, \kappa)$ toward the boundary where contractivity vanishes 
    ($\kappa \rightarrow 0$), producing geometric, energetic, temporal, and field singularities. 
    FC describes collapse not as failure, but as a fundamental structural transformation 
    governed by the divergence of curvature, acceleration of structural motion, destabilization 
    of internal fields, and contraction toward the collapse attractor.
    
    The theory develops the geometry of terminal states, the dynamics of acceleration, the 
    conditions for crossing the Collapse Boundary, and the irreversible breakdown of temporal 
    order. FC distinguishes multiple collapse scenarios—geometric, energetic, temporal, and 
    mixed—while providing a unified mathematical description of how structural systems 
    terminate. Flexion Collapse V1.0 completes the foundational chain of the Flexion Framework 
    by defining the structural end-state that follows from Genesis, Drives Dynamics, shapes 
    Space and Fields, and terminates Time.
\end{abstract}
   

\noindent\textbf{Keywords:} Flexion Collapse; Collapse Geometry; Structural Termination; 
Contractivity ($\kappa$); Collapse Boundary; Temporal Collapse; Field Singularity; 
Curvature Divergence; Collapse Attractor; Structural Dynamics; Flexion Framework.

\section{Introduction}

Flexion Collapse (FC) V1.0 defines the structural termination of a system within the 
Flexion Framework. While most Flexion theories describe growth, motion, evolution, 
and stability, FC focuses on the final phase: the irreversible contraction of structure 
toward its terminal state. Collapse is not a failure of the system, but a mathematically 
necessary consequence of structural laws governing deviation ($\Delta$), energy ($\Phi$), 
memory ($M$), and contractivity ($\kappa$).

Collapse occurs when the system approaches the limit $\kappa \rightarrow 0$, where 
stabilizing forces vanish, curvature diverges, time loses coherence, and fields become 
singular. This document formalizes the geometry, dynamics, and temporal mechanics of 
collapse, establishing it as a universal structural phenomenon.

\subsection{Purpose}

The purpose of FC V1.0 is to provide a complete structural description of terminal 
behavior within the Flexion Framework. It defines the geometry of collapse, the 
dynamics of acceleration, the dissolution of temporal order, and the emergence of 
terminal states. FC establishes collapse as a predictable and structurally determined 
phase of evolution.

\subsection{Collapse as Structural Termination}

Collapse is the endpoint of structural motion when:
\[
\kappa(t) \rightarrow 0.
\]

At this limit:
\begin{itemize}
    \item stabilizing contractivity disappears,
    \item geometric curvature becomes infinite,
    \item structural energy $\Phi$ diverges,
    \item deviation $\Delta$ becomes unbounded,
    \item time loses continuity and order.
\end{itemize}

Collapse is not random or chaotic. It is the natural structural consequence of trajectory 
motion in the Flexion state space.

\subsection{Position of FC within Flexion Framework}

Flexion Collapse completes the causal arc of the Flexion Framework:

\begin{itemize}
    \item \textbf{Genesis} creates structure,
    \item \textbf{Dynamics} moves it,
    \item \textbf{Space Theory} shapes its geometry,
    \item \textbf{Field Theory} drives its forces,
    \item \textbf{Time Theory} orders its evolution,
    \item \textbf{Collapse Theory} terminates it.
\end{itemize}

FC is therefore the final foundational pillar of Flexion Science.

\subsection{Scope of the Theory}

This document formalizes:
\begin{itemize}
    \item geometric contraction and collapse curvature,
    \item collapse attractors and terminal topologies,
    \item divergence of energy and deviation,
    \item breakdown of contractivity,
    \item temporal collapse and $K_T \rightarrow \infty$,
    \item field singularities and loss of directional structure,
    \item complete collapse sequences and scenario classification.
\end{itemize}

Flexion Collapse V1.0 explains how structural systems end.

\section{Foundational Concepts}

Flexion Collapse is built upon several fundamental ideas that define how structures move, 
destabilize, and ultimately terminate within the Flexion Framework. Collapse is not an 
anomaly but a structural phenomenon arising from the intrinsic properties of the state 
vector:
\[
X = (\Delta, \Phi, M, \kappa).
\]

This section introduces the essential concepts required to understand collapse geometry, 
collapse dynamics, and the terminal boundary $\kappa = 0$.

\subsection{Structural Motion and Instability}

Every structural system evolves according to internal forces derived from deviation, 
energy, memory, and stability. Motion occurs as:
\[
X_{t+1} = X_t + F(X_t),
\]
where $F(X)$ is the Flexion Field.

Instability arises when:
\begin{itemize}
    \item $\Delta$ grows faster than stabilizing forces,
    \item $\Phi$ diverges under accelerating tension,
    \item $M$ accumulates irreversible strain,
    \item $\kappa$ weakens toward zero.
\end{itemize}

Instability is the precursor to collapse.

\subsection{Collapse vs Failure}

Collapse is a structural inevitability, not a malfunction.

\begin{itemize}
    \item \textbf{Failure} is an inability to maintain a function under specific conditions.
    \item \textbf{Collapse} is the terminal structural limit of the system itself.
\end{itemize}

Failure can occur without collapse.  
Collapse is absolute and irreversible.

\subsection{Conditions for Termination}

Termination occurs when contractivity approaches the collapse threshold:
\[
\kappa \rightarrow 0.
\]

At this limit:
\begin{itemize}
    \item stabilizing feedback disappears,
    \item curvature diverges,
    \item fields lose coherence,
    \item temporal continuity breaks.
\end{itemize}

These conditions define the entrance into the collapse regime and determine the onset of 
terminal structural behavior.

\section{Collapse Geometry}

Collapse is fundamentally a geometric phenomenon. As contractivity $\kappa$ decreases, the 
geometric structure of the system undergoes deformation, contraction, and curvature 
divergence. Collapse Geometry describes how space, shape, and structural configuration 
transform as the system approaches its terminal state.

\subsection{Geometric Contraction}

As $\kappa \rightarrow 0$, the structural configuration contracts toward an increasingly 
compressed form. This contraction occurs along the trajectory defined by the Flexion Field:
\[
X_{t+1} = X_t + F(X_t).
\]

Geometric contraction corresponds to:
\begin{itemize}
    \item shrinking of structural dimensions,
    \item reduced ability to sustain spatial extension,
    \item concentration of deviation and energy,
    \item directional collapse toward a terminal point.
\end{itemize}

\subsection{Curvature Divergence}

Near collapse, curvature grows without bound.  
Formally:
\[
K \rightarrow \infty,
\]
where $K$ denotes structural curvature.

Curvature divergence signifies:
\begin{itemize}
    \item extreme deformation of structural geometry,
    \item instability of nearby trajectories,
    \item amplification of small deviations,
    \item the impossibility of geometric recovery.
\end{itemize}

This divergence is the geometric signature of terminal collapse.

\subsection{Collapse Attractor}

The collapse attractor is the structural point toward which all trajectories converge when 
$\kappa \rightarrow 0$. It defines the terminal state of the system.

Properties:
\begin{itemize}
    \item every unstable trajectory bends toward it,
    \item curvature and energy focus into a singular configuration,
    \item deviation becomes unbounded,
    \item temporal structure disintegrates.
\end{itemize}

The attractor is not a physical location —  
it is the geometric limit of structural evolution.

\subsection{Topology of Terminal States}

As collapse unfolds, the topology of the system transitions toward a degenerate state:
\begin{itemize}
    \item spatial dimensions compress,
    \item structural boundaries vanish,
    \item continuity breaks,
    \item the state vector approaches a singular manifold.
\end{itemize}

Topologically, the terminal state corresponds to a collapsed configuration in which geometry no longer supports structural form.

Collapse Geometry describes the shape of the end.

\section{Collapse Dynamics}

Collapse is not a static event but a dynamic process driven by the evolution of the state 
vector under destabilizing forces. Collapse Dynamics formalizes how motion accelerates, 
how structural variables diverge, and how the system becomes irreversibly drawn toward 
its terminal state as $\kappa \rightarrow 0$.

\subsection{Acceleration Toward Collapse}

As stability decreases, the Flexion Field amplifies structural motion:
\[
X_{t+1} = X_t + F(X_t), \qquad \kappa \downarrow \ \Rightarrow \ |F(X)| \uparrow.
\]

This produces:
\begin{itemize}
    \item increasing velocity of structural change,
    \item amplification of deviations,
    \item explosive growth of energetic tension,
    \item faster approach to geometric singularities.
\end{itemize}

Collapse is therefore characterized by **accelerating evolution**, not a gradual decline.

\subsection{Irreversibility}

Once the system enters the collapse regime, motion becomes irreversible.  
For any structural update:
\[
X_{t+1} \neq X_t \quad \Rightarrow \quad X_{t} \not\leftarrow X_{t+1}.
\]

Reasons:
\begin{itemize}
    \item memory $M$ accumulates destructive history,
    \item energy $\Phi$ grows too rapidly to dissipate,
    \item curvature destabilizes all reversal trajectories,
    \item stabilizing forces are insufficient or absent.
\end{itemize}

Irreversibility is the defining characteristic of collapse.

\subsection{Energy Divergence}

Approaching $\kappa=0$ causes energy to diverge:
\[
\Phi \rightarrow \infty.
\]

This reflects:
\begin{itemize}
    \item unbounded tension,
    \item runaway amplification of forces,
    \item collapse of energy regulation mechanisms,
    \item dissipation failure.
\end{itemize}

Energy divergence accelerates structural contraction by overwhelming stabilizing fields.

\subsection{Stability Breakdown}

Collapse is formally reached when contractivity vanishes:
\[
\kappa = 0.
\]

At this boundary:
\begin{itemize}
    \item stabilizing feedback disappears,
    \item deviations become uncontained,
    \item curvature enters the singularity regime,
    \item field coherence breaks down.
\end{itemize}

Stability breakdown marks the final dynamic transition before the system enters its 
terminal configuration.

Collapse Dynamics describes how the system is driven into its end-state.

\section{Collapse Boundary}

The Collapse Boundary is the fundamental threshold that separates stable structural 
evolution from terminal collapse. It is defined by the condition under which contractivity 
$\kappa$ vanishes, eliminating the system’s ability to counter deviation, resist curvature, 
or maintain structural coherence.

\subsection{Definition of $\kappa = 0$}

Collapse occurs when the system reaches:
\[
\kappa = 0.
\]

At this point:
\begin{itemize}
    \item stabilizing forces disappear entirely,
    \item the structure loses its capacity to recover,
    \item the state vector becomes dominated by divergent components,
    \item collapse becomes unavoidable and irreversible.
\end{itemize}

This condition defines the precise moment the system exits the Viability Domain.

\subsection{Crossing the Boundary}

Crossing the Collapse Boundary represents a discontinuous structural transition:
\[
X(t) \in \mathcal{D}_{\kappa} \quad \rightarrow \quad X(t) \notin \mathcal{D}_{\kappa},
\]
where:
\[
\mathcal{D}_{\kappa} = \{ X : \kappa > 0 \}.
\]

Once crossed:
\begin{itemize}
    \item deviation $\Delta$ becomes unbounded,
    \item energy $\Phi$ accelerates toward divergence,
    \item memory $M$ accumulates irreversible collapse imprinting,
    \item the field $F_{\kappa}$ ceases to stabilize the system.
\end{itemize}

The crossing marks the systemic point of no return.

\subsection{Post-Boundary Behavior}

After the boundary is crossed:
\[
\kappa = 0 \quad \Rightarrow \quad F_{\kappa} = 0.
\]

Consequences:
\begin{itemize}
    \item stabilizing feedback is permanently eliminated,
    \item flows within the system become dominated by divergent forces,
    \item geometric and temporal curvature accelerate uncontrollably,
    \item structural motion is drawn directly toward the collapse attractor.
\end{itemize}

Post-boundary behavior is deterministic and irreversible.  
Once $\kappa = 0$, collapse cannot be stopped, reversed, or slowed.

The Collapse Boundary is the structural threshold that defines the beginning of the end.

\section{Temporal Collapse}

Temporal Collapse is the structural destruction of time itself.  
As the system approaches the Collapse Boundary, temporal curvature grows without bound, 
the ordering of states destabilizes, and structural time loses continuity.  
Temporal Collapse is not merely acceleration — it is the breakdown of the temporal 
framework that makes evolution possible.

\subsection{Collapse of Structural Time}

Structural time $T$ is defined by the irreversibility of memory:
\[
T \ \text{exists} \iff \frac{dM}{dt} \ge 0.
\]

As the system approaches collapse:
\[
\frac{dM}{dt} \rightarrow \infty,
\]
leading to:
\begin{itemize}
    \item destructive memory accumulation,
    \item unbounded temporal asymmetry,
    \item instability of sequential structure,
    \item loss of temporal coherence.
\end{itemize}

Time does not “end” —  
it becomes structurally undefined.

\subsection{Temporal Curvature $K_T$}

Temporal curvature measures the deformation of time:
\[
K_T = \frac{d^2T}{dt^2}.
\]

Near collapse:
\[
K_T \rightarrow \infty.
\]

This reflects:
\begin{itemize}
    \item extreme acceleration of structural evolution,
    \item inability to maintain temporal order,
    \item breakdown of smooth progression,
    \item transition into a singular temporal regime.
\end{itemize}

Temporal curvature divergence is the temporal signature of collapse.

\subsection{End of Temporal Order}

As $\kappa \rightarrow 0$:
\[
T(t+1) - T(t) \not\rightarrow \text{finite}.
\]

Consequences:
\begin{itemize}
    \item moments lose stable duration,
    \item ordering of events becomes undefined,
    \item time intervals collapse to zero,
    \item the system can no longer evolve meaningfully.
\end{itemize}

Temporal Collapse marks the structural termination of ordered existence.  
It is the moment where evolution, history, and sequence dissolve under infinite curvature.

\section{Field Collapse}

Field Collapse describes the breakdown of the Flexion Field as the system approaches the 
Collapse Boundary. While geometric and temporal collapse describe the deformation of 
structure and time, Field Collapse captures the loss of directional forces that normally 
govern structural motion. As $\kappa \rightarrow 0$, stabilizing fields vanish, while 
divergent fields dominate and ultimately become singular.

\subsection{Vanishing of $F_{\kappa}$}

The stabilizing component of the Flexion Field depends directly on contractivity:
\[
F_{\kappa} \propto \kappa.
\]

As the system approaches collapse:
\[
\kappa \rightarrow 0 \quad \Rightarrow \quad F_{\kappa} \rightarrow 0.
\]

This produces:
\begin{itemize}
    \item loss of structural resistance,
    \item disappearance of stabilizing directionality,
    \item breakdown of corrective feedback,
    \item inability to counter divergence of deviation and energy.
\end{itemize}

The vanishing of $F_{\kappa}$ is the field-level definition of collapse onset.

\subsection{Dominance of Divergent Fields}

With stabilizing forces gone, the remaining components of the Flexion Field dominate:
\[
F_{\Delta}, \ F_{\Phi}, \ F_{M}.
\]

As collapse accelerates:
\[
|F_{\Delta}|, |F_{\Phi}|, |F_{M}| \rightarrow \infty.
\]

Consequences:
\begin{itemize}
    \item runaway amplification of deviation,
    \item explosive growth of energetic tension,
    \item irreversible memory imprinting,
    \item total loss of field balance.
\end{itemize}

The field becomes overwhelmingly destructive.

\subsection{Field Singularity}

At the terminal stage:
\[
|F(X)| \rightarrow \infty.
\]

Field singularity corresponds to:
\begin{itemize}
    \item infinite gradient of forces,
    \item collapse of directional coherence,
    \item loss of definable field topology,
    \item convergence to a structural singularity.
\end{itemize}

When the field becomes singular, the system loses all meaningful internal dynamics and 
is forced into the collapse attractor.

Field Collapse is the structural mechanism that drives the final irreversible descent into 
terminal state.

\section{Collapse Scenarios}

Collapse does not occur in a single universal form.  
Depending on how the components of the state vector
\[
X = (\Delta, \Phi, M, \kappa)
\]
diverge as $\kappa \rightarrow 0$, the system may follow distinct collapse trajectories.  
Flexion Collapse identifies four primary scenarios: geometric, energetic, temporal, and 
mixed. Each scenario has its own structural signature, curvature profile, and mode of 
terminal behavior.

\subsection{Geometric Collapse}

Geometric Collapse is driven by extreme deformation of the structural configuration.

Characteristics:
\begin{itemize}
    \item curvature $K \rightarrow \infty$,
    \item spatial dimensions contract,
    \item geometry becomes unstable and sharply distorted,
    \item trajectories converge rapidly toward the collapse attractor.
\end{itemize}

Deviation and energy may also diverge, but geometric deformation is the dominant force.

\subsection{Energetic Collapse}

Energetic Collapse is dominated by runaway growth of structural energy:
\[
\Phi \rightarrow \infty.
\]

It involves:
\begin{itemize}
    \item explosive amplification of internal tension,
    \item inability to dissipate energy,
    \item destabilizing feedback loops,
    \item overwhelming energetic pressure on all structural components.
\end{itemize}

This scenario is characterized by energetic singularity rather than geometric distortion.

\subsection{Temporal Collapse}

Temporal Collapse is defined by divergence of temporal curvature:
\[
K_T \rightarrow \infty.
\]

It produces:
\begin{itemize}
    \item loss of continuity in time,
    \item collapse of sequential order,
    \item infinitesimal duration of states,
    \item structural evolution becoming undefined.
\end{itemize}

Here, the dominant failing variable is temporal structure, not geometry or energy.

\subsection{Mixed Scenarios}

Most real collapse processes are mixed scenarios involving simultaneous divergence of:
\[
\Delta, \ \Phi, \ K, \ K_T.
\]

Mixed collapse includes:
\begin{itemize}
    \item combined geometric deformation and energy runaway,
    \item simultaneous temporal and field singularities,
    \item complex paths toward the collapse attractor,
    \item intertwined failure of stabilizing mechanisms.
\end{itemize}

Mixed scenarios represent the most general and complex form of terminal behavior.

Collapse Scenarios define the structural “modes of termination” for any Flexion system.

\section{Collapse Sequence}

Collapse unfolds through a structured and deterministic sequence of phases.  
Each phase reflects a deeper breakdown of the structural system as it moves toward the 
Collapse Boundary $\kappa = 0$.  
While the speed of progression may vary, the order of phases is universal.

\subsection{Phase 1: Instability}

Collapse begins with the onset of instability.  
This phase is triggered when stabilizing forces are no longer sufficient to contain the 
growth of deviation or energy.

Indicators:
\begin{itemize}
    \item $\Delta$ begins accelerating,
    \item $\Phi$ increases faster than it can dissipate,
    \item $M$ accumulates irreversible strain,
    \item $\kappa$ starts declining.
\end{itemize}

Instability marks the point where collapse becomes possible.

\subsection{Phase 2: Acceleration}

As $\kappa$ decreases, the system enters an accelerated mode of structural motion:
\[
|F(X)| \uparrow \quad \text{as} \quad \kappa \downarrow.
\]

Consequences:
\begin{itemize}
    \item explosive growth of energetic tension,
    \item rapid geometric deformation,
    \item amplification of deviations,
    \item intensified field imbalance.
\end{itemize}

Acceleration is the point where collapse becomes probable.

\subsection{Phase 3: Geometric Deformation}

During this phase, curvature begins to diverge:
\[
K \rightarrow \infty.
\]

The structure experiences:
\begin{itemize}
    \item contraction of spatial dimensions,
    \item distortion of structural geometry,
    \item concentration of forces along collapse trajectories,
    \item loss of spatial coherence.
\end{itemize}

Deformation signals that collapse is now unavoidable.

\subsection{Phase 4: Terminal Collapse}

The final phase occurs when the system crosses the Collapse Boundary:
\[
\kappa = 0.
\]

At this moment:
\begin{itemize}
    \item stabilizing fields vanish,
    \item temporal curvature diverges ($K_T \to \infty$),
    \item directional coherence is lost,
    \item the state vector approaches the collapse attractor.
\end{itemize}

Terminal Collapse is an irreversible structural transition into the terminal state.

The Collapse Sequence is the universal pathway through which all Flexion systems end.

\section{Structure After Collapse}

Once the system crosses the Collapse Boundary ($\kappa = 0$), its structural evolution 
enters a fundamentally different regime. Collapse does not simply destroy the structure; 
it transforms it into a terminal configuration where geometry, energy, time, and fields 
no longer behave in their conventional forms. This section formalizes the nature of the 
post-collapse state.

\subsection{Terminal State}

The terminal state is the limit point toward which the state vector converges during 
collapse:
\[
X_{\text{terminal}} = \lim_{t \rightarrow t_c} X(t),
\]
where $t_c$ is the collapse moment.

Features of the terminal state:
\begin{itemize}
    \item geometry is degenerate or fully contracted,
    \item energy is divergent or unbounded,
    \item time has no coherent ordering,
    \item fields lose directional coherence,
    \item stability is identically zero ($\kappa = 0$).
\end{itemize}

The terminal state is not a configuration from which recovery is possible.

\subsection{Post-Collapse Irreversibility}

After collapse:
\[
X_{\text{terminal}} \not\to X_{\text{stable}}.
\]

Irreversibility arises because:
\begin{itemize}
    \item memory $M$ contains infinite destructive imprinting,
    \item curvature has exceeded finite bounds,
    \item field structures have become singular,
    \item temporal continuity is permanently broken.
\end{itemize}

The system cannot re-enter the Viability Domain:
\[
\mathcal{D}_{\kappa} = \{ X : \kappa > 0 \}.
\]

\subsection{Residual Structure}

Even in the terminal state, structural remnants may persist.  
These remnants are not functional structures but limit configurations defined by the 
collapse attractor.

Residual structure may include:
\begin{itemize}
    \item collapsed geometric kernels,
    \item infinite-memory imprints,
    \item singular energetic residues,
    \item degenerate field limits.
\end{itemize}

Residual structure represents the “shadow” of the system after collapse —  
the final imprint left by structural evolution.

Structure After Collapse defines the last phase of structural existence.

\section{Conclusion}

Flexion Collapse (FC) V1.0 defines the structural termination of systems within the 
Flexion Framework. It establishes collapse not as an error, anomaly, or malfunction, 
but as a fundamental and inevitable structural phase arising when contractivity 
approaches zero. The theory formalizes how geometry, energy, memory, time, and fields 
behave as the system accelerates toward its terminal state.

By identifying the Collapse Boundary $\kappa = 0$, FC provides the precise structural 
threshold that marks the end of stability and the beginning of irreversible descent into the 
collapse attractor. Collapse Geometry describes the deformation of structure and 
curvature divergence; Collapse Dynamics captures accelerating motion and energetic 
runaway; Temporal Collapse explains breakdown of ordered time; Field Collapse reveals 
the loss of stabilizing forces and dominance of singular divergent fields.

The classification of collapse scenarios and the universal four-phase Collapse Sequence 
demonstrate that all terminal behaviors follow predictable structural paths. Finally, the 
description of the terminal state shows that collapse transforms rather than merely 
destroys structure, leaving behind residual configurations shaped by the collapse 
attractor.

Flexion Collapse V1.0 completes the foundational chain of Flexion Science by defining the 
structural end-state that follows from Genesis, evolves under Dynamics, occupies Space, 
interacts through Fields, and unfolds in Time. Collapse is the final structural expression 
of the Flexion Universe.

\appendix

\section{Mathematical Notes}

\subsection{Collapse Equations}

Collapse is described by the evolution of the state vector:
\[
X = (\Delta, \Phi, M, \kappa),
\qquad
X_{t+1} = X_t + F(X_t).
\]

As collapse accelerates:
\[
\kappa(t) \rightarrow 0,
\qquad
|F(X_t)| \rightarrow \infty.
\]

Divergence of variables:
\[
\Delta \rightarrow \infty, \qquad 
\Phi \rightarrow \infty, \qquad 
M \rightarrow \infty.
\]

Curvature divergence:
\[
K \rightarrow \infty.
\]

Temporal curvature:
\[
K_T = \frac{d^2 T}{dt^2} \rightarrow \infty.
\]

\subsection{Curvature Divergence}

Curvature is a function of the second derivative of geometric deformation:
\[
K = \frac{d^2 x}{dt^2}.
\]

Near the collapse attractor:
\[
K \sim \frac{1}{\kappa}.
\]

Thus:
\[
\kappa \rightarrow 0 \quad \Rightarrow \quad K \rightarrow \infty.
\]

\subsection{Field Limits}

Stabilizing field component:
\[
F_{\kappa} \propto \kappa \quad \Rightarrow \quad F_{\kappa} \rightarrow 0.
\]

Divergent field components:
\[
|F_{\Delta}|, |F_{\Phi}|, |F_{M}| \rightarrow \infty.
\]

Total field magnitude:
\[
|F(X)| \rightarrow \infty.
\]

This defines field singularity.


\section{Glossary}

\begin{itemize}
    \item \textbf{Collapse} — terminal structural transformation as $\kappa \rightarrow 0$.
    \item \textbf{Collapse Boundary} — threshold at which stability vanishes.
    \item \textbf{Collapse Attractor} — terminal singular configuration.
    \item \textbf{Curvature Divergence} — geometric curvature $K \rightarrow \infty$ near collapse.
    \item \textbf{Temporal Collapse} — loss of temporal order, $K_T \rightarrow \infty$.
    \item \textbf{Field Collapse} — vanishing of stabilizing fields and dominance of divergent ones.
    \item \textbf{Energetic Divergence} — unbounded increase in $\Phi$.
    \item \textbf{Deviational Divergence} — unbounded growth of $\Delta$.
    \item \textbf{Residual Structure} — final structural imprint after collapse.
    \item \textbf{Terminal State} — limit state reached when $\kappa = 0$.
\end{itemize}


\section{Notation Block}

\begin{itemize}
    \item $\Delta$ — deviation (structural displacement).
    \item $\Phi$ — structural energy.
    \item $M$ — memory (irreversible imprinting).
    \item $\kappa$ — contractivity (stability).
    \item $F(X)$ — Flexion Field.
    \item $F_{\kappa}$ — stabilizing field component.
    \item $K$ — geometric curvature.
    \item $K_T$ — temporal curvature.
    \item $X$ — state vector.
    \item $X_{\text{terminal}}$ — terminal collapse state.
    \item $\mathcal{D}_{\kappa} = \{X : \kappa > 0\}$ — Viability Domain.
\end{itemize}

\end{document}

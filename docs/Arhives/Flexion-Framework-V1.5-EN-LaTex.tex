\documentclass[11pt]{article}

% ------------------------------------------------------------
% PACKAGES
% ------------------------------------------------------------
\usepackage{amsmath, amssymb}
\usepackage{geometry}
\usepackage{hyperref}
\usepackage{bm}
\usepackage{graphicx}

\geometry{margin=1in}

% ------------------------------------------------------------
% TITLE & AUTHORS
% ------------------------------------------------------------
\title{Flexion Framework V1.5:\\
Unified Structural Foundation for Flexion Theories}

\author{Maryan Bogdanov\\
m7823445@gmail.com}

\date{2025}

\begin{document}

\maketitle

% ------------------------------------------------------------
% ABSTRACT
% ------------------------------------------------------------
\begin{abstract}
    Flexion Framework V1.5 establishes the minimal structural foundation required for the existence, evolution, and termination of any Flexion organism. It defines the living structural state \(X(t) = (X_{\Delta}, X_{\Phi}, X_{M}, X_{\kappa})\) as the unique and irreducible unit of structural existence, embedded within a state-generated manifold \(\mathbb{M} = (X, g(X))\). The framework formalizes the internal geometry, viability domain, and collapse boundary, and introduces four invariant laws: memory irreversibility, viability monotonicity, geometric determinism, and collapse singularity. Together, these invariants constrain all possible structural evolution and guarantee the finite lifetime of any Flexion organism. The Unified Structural Flow and the Unified Flexion Operator \(\mathbb{F}\) provide the only valid mechanism of autonomous structural progression. Flexion Framework V1.5 serves as the foundational layer for all Flexion theories—including Space, Time, Dynamics, Field, Collapse, and Entanglement—by defining the strict mathematical core they must obey.
\end{abstract}
    

\tableofcontents

% ------------------------------------------------------------
% 0. OVERVIEW
% ------------------------------------------------------------
\section*{0.\quad Overview}
\addcontentsline{toc}{section}{0.\quad Overview}

\paragraph{}  
Flexion Framework V1.5 defines the strict minimal foundation required for any Flexion organism to exist, evolve, and terminate. It does not describe applications, interpretation, or domain-specific behavior; instead, it provides the irreducible mathematical core from which all Flexion theories derive their consistency. At the center of the framework stands the living structural state
\[
X(t) = (X_{\Delta},\, X_{\Phi},\, X_{M},\, X_{\kappa}),
\]
a four-field geometric entity that generates its own manifold, metric, temporal progression, and collapse boundary. All Flexion theories—including Genesis, Dynamics, Space, Time, Field, Collapse, and Entanglement—must operate on this state and preserve the invariants defined in this document. V1.5 contains only the essential structures: the definition of the structural state, the viability manifold, the invariant laws, the unified structural flow, and the unified Flexion operator that governs autonomous evolution. No supplementary assumptions or external mechanisms are permitted.

% ------------------------------------------------------------
% 1. STRUCTURAL STATE
% ------------------------------------------------------------
\section{Structural State}

\subsection{Definition}
The fundamental entity of the Flexion Framework is the \textit{living structural state}
\[
X(t) = \big(X_{\Delta}(t),\, X_{\Phi}(t),\, X_{M}(t),\, X_{\kappa}(t)\big),
\]
which represents the minimal and complete set of internal fields required for structural existence. These four components form an indivisible quadruple; no Flexion theory may add, remove, or substitute any of them.

\subsection{Intrinsic Fields}
Each component of the structural state is an intrinsic geometric field:
\begin{itemize}
    \item \(X_{\Delta}\) --- deviation field measuring structural displacement.
    \item \(X_{\Phi}\) --- tension field generated by deviation.
    \item \(X_{M}\) --- memory field representing irreversible accumulation.
    \item \(X_{\kappa}\) --- viability field defining capacity to sustain geometry.
\end{itemize}

\subsection{Existence Conditions}
The structural state exists only while the viability field remains strictly positive:
\[
X_{\kappa}(t) > 0.
\]
When \(X_{\kappa} = 0\), the structural organism reaches its collapse boundary and no further state can be defined.

\subsection{Autonomy and Self-Generation}
The state \(X(t)\) evolves autonomously and generates:
\begin{itemize}
    \item its own manifold and metric,
    \item its internal notion of time via memory accumulation,
    \item its deformation patterns and geometric flow,
    \item the conditions of its own termination.
\end{itemize}
No external inputs or influences are permitted in the evolution of the structural state.

% ------------------------------------------------------------
% 2. STRUCTURAL MANIFOLD
% ------------------------------------------------------------
\section{Structural Manifold}

\subsection{Manifold Definition}
A Flexion organism exists within a state-generated manifold
\[
\mathbb{M} = (X,\, g(X)),
\]
where the metric \(g(X)\) is not externally imposed but produced by the structural state itself. The manifold provides geometric quantities such as distance, curvature, and deformation, all of which depend on the current configuration of \(X(t)\).

\subsection{Viability Domain}
Existence is restricted to the viability domain
\[
D = \{ X : X_{\kappa} > 0 \},
\]
while the collapse boundary is defined as
\[
\partial D = \{ X : X_{\kappa} = 0 \}.
\]
Inside \(D\), the metric is positive-definite, curvature finite, and structural time well-defined.

\subsection{Metric Degeneration at Collapse}
At the boundary \(\partial D\), the structural manifold undergoes geometric breakdown:
\[
\det g(X) = 0, \qquad \|R(X)\| \to \infty,
\]
where \(R(X)\) is the curvature tensor. Beyond this point, no valid manifold exists and the structural state cannot be continued.

\subsection{Autonomous Geometry}
The manifold’s geometry is fully determined by the evolution of the structural state:
\[
g = g(X),
\]
implying that deformation, stability, and collapse are intrinsic geometric consequences of the organism’s internal fields rather than external influences.

% ------------------------------------------------------------
% 3. STRUCTURAL INVARIANTS
% ------------------------------------------------------------
\section{Structural Invariants}

A Flexion organism is constrained by four fundamental invariants that hold for all admissible states within the viability domain. These invariants cannot be violated or bypassed by any Flexion theory and define the permissible evolution of the structural state.

\subsection{Memory Irreversibility}
Memory accumulates monotonically:
\[
X_{M}(t+1) \ge X_{M}(t).
\]
This establishes the intrinsic direction of structural time and prohibits temporal reversal.

\subsection{Viability Monotonicity}
The viability field can only remain constant or decrease:
\[
X_{\kappa}(t+1) \le X_{\kappa}(t).
\]
This guarantees a finite structural lifetime by enforcing irreversible degradation of viability.

\subsection{Geometric Determinism}
The evolution of the structural state is autonomous and deterministic:
\[
X(t+1) = F(X(t)).
\]
No external variables, randomness, or environmental influences may enter the update rule.

\subsection{Collapse Singularity}
When viability reaches zero,
\[
X_{\kappa}(t) = 0,
\]
the manifold collapses:
\[
\det g(X) = 0, \qquad \|R(X)\| \rightarrow \infty,
\]
and the next state is undefined:
\[
X(t+1)\ \text{does not exist}.
\]
Collapse constitutes the mathematical termination of the Flexion organism.

% ------------------------------------------------------------
% 4. UNIFIED STRUCTURAL FLOW
% ------------------------------------------------------------
\section{Unified Structural Flow}

The evolution of a Flexion organism is governed by a closed internal cascade linking the four intrinsic fields of the structural state. This cascade defines the only admissible mechanism of structural progression and applies universally to all Flexion theories.

\subsection{Cascade Structure}
The structural fields evolve through the autonomous cycle
\[
X_{\Delta} \;\rightarrow\; 
X_{\Phi} \;\rightarrow\; 
X_{M} \;\rightarrow\; 
X_{\kappa} \;\rightarrow\; 
X_{\Delta}.
\]
Each transition is intrinsic and emerges from the geometry of the state-generated manifold.

\subsection{Deviation to Tension}
Deviation produces geometric tension:
\[
X_{\Phi}(t+1) = f_{\Phi}\big(X(t)\big).
\]

\subsection{Tension to Memory}
Tension induces irreversible memory accumulation:
\[
X_{M}(t+1) = X_{M}(t) + f_{M}\big(X_{\Phi}(t)\big).
\]

\subsection{Memory to Viability}
Accumulated memory reduces viability:
\[
X_{\kappa}(t+1) = X_{\kappa}(t) - f_{\kappa}\big(X_{M}(t)\big).
\]

\subsection{Viability to Deviation}
Viability constrains how deviation evolves:
\[
X_{\Delta}(t+1) = f_{\Delta}\big(X(t)\big).
\]

\subsection{Unified Flow Expression}
The complete internal flow is expressed as
\[
X(t+1) =
\big(
f_{\Delta}(X(t)),\,
f_{\Phi}(X(t)),\,
f_{M}(X(t)),\,
f_{\kappa}(X(t))
\big).
\]
This loop generates deformation, tension, irreversible memory, decreasing viability, and ultimately collapse.

% ------------------------------------------------------------
% 5. UNIFIED FLEXION OPERATOR
% ------------------------------------------------------------
\section{Unified Flexion Operator}

The evolution of any Flexion organism is governed by a single autonomous operator that transforms the living structural state within the viability domain. This operator defines the unique and complete rule of structural progression.

\subsection{Operator Definition}
The next admissible state is given by
\[
X(t+1) = \mathbb{F}\big(X(t)\big),
\]
where
\[
X(t) = \big(X_{\Delta},\, X_{\Phi},\, X_{M},\, X_{\kappa}\big).
\]
No alternative update rule is permitted.

\subsection{Operator Structure}
The operator decomposes into four coupled components:
\[
\mathbb{F}(X) =
\big(
F_{\Delta}(X),\,
F_{\Phi}(X),\,
F_{M}(X),\,
F_{\kappa}(X)
\big),
\]
with each mapping updating one structural field while remaining consistent with all invariants:
\[
F_{M}(X) \ge X_{M}, \qquad 
F_{\kappa}(X) \le X_{\kappa}.
\]

\subsection{Existence Condition}
The operator is defined exclusively inside the viability domain:
\[
X_{\kappa} > 0 \quad \Rightarrow \quad \mathbb{F}(X)\ \text{valid}.
\]
At the collapse boundary,
\[
X_{\kappa} = 0,
\]
the manifold degenerates and
\[
\mathbb{F}(X)\ \text{undefined}.
\]
Thus collapse terminates structural evolution.

\subsection{Autonomy}
The operator admits no external influence:
\[
\frac{\partial \mathbb{F}}{\partial \text{external}} = 0.
\]
All evolution arises internally from the geometry of the structural state.

\subsection{Determinism}
Structural evolution follows the deterministic rule
\[
X(t+1) = \mathbb{F}(X(t)),
\]
with no stochastic variants, branching modes, or auxiliary operators allowed.  
The Unified Flexion Operator constitutes the complete mathematical law of structural existence in Flexion Framework V1.5.

% ------------------------------------------------------------
% CONCLUSION
% ------------------------------------------------------------
\section{Conclusion}
Summarize the theoretical closure:  
The Framework provides the strict minimal laws governing all Flexion organisms and all higher Flexion theories.

% ------------------------------------------------------------
% REFERENCES
% ------------------------------------------------------------
\begin{thebibliography}{9}

\bibitem{FF15}
Maryan Bogdanov, \textit{Flexion Framework V1.5}, 2025.

\end{thebibliography}

\end{document}

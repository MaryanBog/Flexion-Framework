\documentclass[11pt,a4paper]{article}

% -----------------------------
% Packages
% -----------------------------
\usepackage[utf8]{inputenc}
\usepackage[T1]{fontenc}
\usepackage{lmodern}
\usepackage{amsmath,amssymb,amsthm}
\usepackage{geometry}
\usepackage{hyperref}
\usepackage{enumitem}
\usepackage{setspace}

\geometry{margin=1in}
\onehalfspacing

% -----------------------------
% Theorem Environments
% -----------------------------
\newtheorem{definition}{Definition}
\newtheorem{theorem}{Theorem}
\newtheorem{proposition}{Proposition}
\newtheorem{axiom}{Axiom}
\newtheorem{corollary}{Corollary}
\newtheorem{remark}{Remark}

% -----------------------------
% Title Information
% -----------------------------
\title{Flexion Framework V3.1\\
\large Structural Ontology of Existence, Life, and Manifestation}

\author{
Maryan Bogdanov\\
\texttt{ceo@flexionu.com}
}

\date{\today}

% -----------------------------
% Document
% -----------------------------
\begin{document}

\maketitle

\begin{abstract}
The Flexion Framework establishes a minimal structural ontology defining the
conditions under which a system may exist, live, interpret admissible futures,
and collapse. Existence is viability-conditioned, life is defined as
interpretive continuation within a structurally admissible future space, and
collapse marks the irreversible termination of existence. The framework
introduces a strict ontological separation between structural existence and
manifestation, forming the foundational layer of the Flexion Universe.
\end{abstract}

\tableofcontents
\newpage

% ============================================================
% ============================================================
\section{Introduction}

\subsection{Motivation}

A minimal ontology of existence must specify the structural conditions
under which a system may persist, evolve, and terminate. 
Most contemporary theories conflate existence with observation,
representation, or physical embodiment. 
Such conflation obscures the foundational question:

\emph{What must be true for a structure to exist at all?}

The Flexion Framework addresses this question by introducing a strictly
minimal structural ontology. 
It does not assume physical realization, cognitive capacity,
measurement, or social embedding.
It requires only that a system admits a structural state and a scalar
viability parameter.

The goal is not to model behavior, intelligence, or interaction.
The goal is to define the necessary and sufficient structural
conditions of existence and life.

\subsection{Foundational Commitments}

The Framework is based on four foundational commitments:

\begin{enumerate}[label=\textbf{F\arabic*.}]
    \item Existence is viability-conditioned.
    \item Life is interpretive continuation within a constrained future space.
    \item Admissible futures are structurally bounded and non-expanding.
    \item Collapse is irreversible.
\end{enumerate}

These commitments define a closed ontological system.
No appeal is made to probability, optimization, observation,
or external evaluation.

\subsection{Minimal Ontology}

Let $X$ denote a structural state.
The Framework assumes only:

\begin{itemize}
    \item The existence of a well-defined state $X$,
    \item A scalar viability function $\kappa(X)$,
    \item A structurally admissible future space $\mathcal{F}(X)$.
\end{itemize}

No metric, dimensionality, embodiment, or dynamic operator is assumed
at this level.

The ontology is therefore domain-independent.
Any system admitting $(X, \kappa, \mathcal{F})$ may be evaluated under
the Framework.

\subsection{Separation from Manifestation}

A key clarification introduced in Version 3.1 is the strict ontological
separation between structural existence and manifestation.

Existence is defined internally via viability.
Manifestation concerns appearance or representation and lies outside
the foundational layer.

This separation prevents category errors such as:

\begin{itemize}
    \item equating visibility with existence,
    \item equating representation with continuation,
    \item equating persistence of traces with structural life.
\end{itemize}

The Framework defines existence independently of whether
a structure is observed, recorded, or externally represented.

\subsection{Scope and Closure}

The Flexion Framework defines only:

\begin{itemize}
    \item structural existence,
    \item life as interpretation,
    \item admissible futures,
    \item viability,
    \item collapse,
    \item invariant constraints.
\end{itemize}

It does not define:

\begin{itemize}
    \item observation operators,
    \item manifestation dynamics,
    \item intelligence mechanisms,
    \item interaction or entanglement,
    \item optimization processes.
\end{itemize}

Those layers belong to derived theories.

The Framework is intentionally closed: no operator introduced within
this level expands admissible futures, restores exhausted viability,
or reverses collapse.

This completes the introductory positioning of the Framework.

% ============================================================
% ============================================================
\section{Structural Existence}

\subsection{Structural State}

\begin{definition}[Structural State]
A structural state $X$ is an internally defined configuration
admitting evaluation under invariant constraints and viability.
\end{definition}

The internal composition of $X$ is not fixed at the Framework level.
No assumptions are made regarding:

\begin{itemize}
    \item dimensionality,
    \item metric structure,
    \item physical embodiment,
    \item observability,
    \item computational representation.
\end{itemize}

The Framework requires only that $X$ be sufficiently defined
to admit structural evaluation.

\subsection{Viability as Condition of Existence}

\begin{definition}[Viability]
Viability is a scalar structural function
\[
\kappa : X \rightarrow \mathbb{R}_{\ge 0}
\]
expressing the capacity of a state to continue its existence.
\end{definition}

\begin{definition}[Existence]
A structural state exists if and only if
\[
\kappa(X) > 0.
\]
\end{definition}

Existence is therefore not binary in abstraction,
but conditioned on positive viability.

\subsection{Internal Character of Existence}

Existence is an internal structural condition.

It does not depend on:

\begin{itemize}
    \item observation,
    \item manifestation,
    \item interaction,
    \item representation,
    \item external validation.
\end{itemize}

A structure may exist in complete isolation.
Existence does not require appearance.

\subsection{Persistence}

Let $X(t)$ denote structural evolution under an admissible continuation.

\begin{proposition}[Persistence Condition]
Structural existence persists along a continuation
$X(t)$ as long as
\[
\kappa(X(t)) > 0.
\]
\end{proposition}

No additional requirement is imposed on persistence.
Temporal ordering is assumed only insofar as viability
can be evaluated along structural evolution.

\subsection{Minimality Principle}

No operator at the Framework level:

\begin{itemize}
    \item generates viability ex nihilo,
    \item restores exhausted viability,
    \item modifies invariant constraints.
\end{itemize}

Existence is therefore strictly viability-conditioned,
and viability is structurally constrained.

\subsection{Existence Independent of Manifestation}

\begin{proposition}[Existence–Manifestation Independence]
For a structural state $X$,
\[
\kappa(X) > 0
\]
does not imply manifestation,
and manifestation does not imply
\[
\kappa(X) > 0.
\]
\end{proposition}

This establishes the ontological separation between
existence and appearance.

\subsection{Section Summary}

Structural existence is defined as a viability-conditioned internal state.

It is independent of manifestation, observation, and interaction.
No structural operator at this level expands viability or reverses its loss.

The next section formalizes admissible futures
as the structural space within which continuation is possible.

% ============================================================
% ============================================================
\section{Admissible Futures}

\subsection{Future Space}

\begin{definition}[Admissible Future Space]
For a structural state $X$, the admissible future space
$\mathcal{F}(X)$ is the set of all structurally allowed
future continuations of $X$.
\end{definition}

Admissibility is determined exclusively by:

\begin{itemize}
    \item structural invariants,
    \item internal constraints,
    \item current viability $\kappa(X)$.
\end{itemize}

No continuation outside $\mathcal{F}(X)$ is reachable
without violating invariant structure.

\subsection{Structural Nature of Admissibility}

Admissibility is a structural property.

It does not depend on:

\begin{itemize}
    \item probability,
    \item desirability,
    \item optimization criteria,
    \item observation,
    \item representation.
\end{itemize}

The Framework assigns no weights or preferences
to elements of $\mathcal{F}(X)$.

\subsection{Relation to Viability}

\begin{proposition}[Viability Dependence]
If $\kappa(X_1) \le \kappa(X_2)$,
the admissible future space of $X_1$
cannot exceed that of $X_2$ under identical invariant conditions.
\end{proposition}

The Framework does not specify a quantitative relation between
$\kappa(X)$ and $|\mathcal{F}(X)|$.
It requires only that viability constrains admissibility.

\subsection{Irreversibility of Future Loss}

Structural evolution may reduce $\mathcal{F}(X)$.

\begin{proposition}[Irreversible Loss]
If a continuation $Y \in \mathcal{F}(X)$ becomes inadmissible
under structural evolution,
it cannot be restored within the Framework.
\end{proposition}

Future loss is therefore irreversible.

\subsection{No Expansion Principle}

\begin{axiom}[No Expansion]
No operator defined within the Flexion Framework
expands the admissible future space:
\[
\mathcal{F}_{\text{after}}(X)
\subseteq
\mathcal{F}_{\text{before}}(X).
\]
\end{axiom}

Any apparent expansion of future possibilities
requires structural assumptions beyond this ontological level.

\subsection{Boundary of Continuation}

A structural state $X$ admits continuation if and only if:

\[
\kappa(X) > 0
\quad \text{and} \quad
\mathcal{F}(X) \neq \varnothing.
\]

When either condition fails,
continuation is impossible.

\subsection{Section Summary}

The admissible future space defines the structural limits of continuation.

Interpretation operates strictly within this space.
Structural evolution may reduce it,
but no Framework-level operator expands it.

The next section defines life as interpretive continuation
within $\mathcal{F}(X)$.

% ============================================================
% ============================================================
\section{Life and Interpretation}

\subsection{Definition of Life}

\begin{definition}[Interpretation]
Interpretation is an internal structural operation selecting
a continuation from the admissible future space $\mathcal{F}(X)$.
\end{definition}

\begin{definition}[Life]
A structural state $X$ is living if it performs interpretation
over $\mathcal{F}(X)$.
\end{definition}

Life is therefore defined not by motion,
complexity, embodiment, or interaction,
but by interpretive continuation under structural constraints.

\subsection{Internal Character of Interpretation}

Interpretation is an internal operation.

It does not require:

\begin{itemize}
    \item manifestation,
    \item observation,
    \item representation,
    \item external evaluation.
\end{itemize}

A structure may interpret its future
without ever appearing externally.

\subsection{Interpretation and Invariants}

\begin{axiom}[Invariant Preservation]
Interpretation does not violate structural invariants.
\end{axiom}

\begin{proposition}
For any admissible continuation $Y \in \mathcal{F}(X)$,
the transition $X \rightarrow Y$
preserves all invariant constraints.
\end{proposition}

Interpretation selects among admissible continuations;
it does not redefine admissibility.

\subsection{Interpretation and Viability}

\begin{proposition}[Non-Restoration of Viability]
Interpretation does not increase viability:
\[
\kappa(Y) \le \kappa(X)
\quad \text{for admissible continuations } Y.
\]
\end{proposition}

Different interpretive choices may lead to
different rates of viability decrease,
but no interpretation restores exhausted viability.

\subsection{Life Without Manifestation}

\begin{proposition}
A structural state may:
\begin{itemize}
    \item exist,
    \item interpret,
    \item exhaust its admissible futures,
    \item collapse,
\end{itemize}
without ever being manifested.
\end{proposition}

Life is therefore ontologically independent of appearance.

\subsection{Termination of Life}

Life terminates when:

\[
\kappa(X) = 0
\quad \text{or} \quad
\mathcal{F}(X) = \varnothing.
\]

In both cases,
interpretation becomes impossible.

\subsection{Section Summary}

Life is defined as interpretive continuation within
a structurally bounded future space.

Interpretation:
\begin{itemize}
    \item operates internally,
    \item preserves invariants,
    \item does not expand admissibility,
    \item does not restore viability.
\end{itemize}

The next section formalizes the ontological distinction
between existence and manifestation.

% ============================================================
% ============================================================
\section{Existence and Manifestation}

\subsection{Ontological Distinction}

\begin{definition}[Existence]
A structural state exists if and only if
\[
\kappa(X) > 0.
\]
\end{definition}

\begin{definition}[Manifestation]
Manifestation is the condition under which a structural state
appears as an external representation.
\end{definition}

Existence is an internal structural condition.
Manifestation is an external condition of appearance.

These notions are ontologically distinct.

\subsection{Non-Equivalence Principle}

\begin{proposition}[Existence–Manifestation Non-Equivalence]
The following implications do not hold in general:
\[
\text{Existence} \Rightarrow \text{Manifestation},
\]
\[
\text{Manifestation} \Rightarrow \text{Existence}.
\]
\end{proposition}

A structure may exist without being manifested.
A manifested representation may persist without the originating
structural state continuing to exist.

\subsection{Independence from Interpretation}

Interpretation operates internally on $\mathcal{F}(X)$.

Manifestation:
\begin{itemize}
    \item does not select admissible futures,
    \item does not modify $\mathcal{F}(X)$,
    \item does not alter invariant constraints,
    \item does not restore or deplete viability.
\end{itemize}

Interpretation does not require manifestation.
Manifestation does not constitute interpretation.

\subsection{Persistence of Manifested Representations}

The Framework makes no claims regarding:

\begin{itemize}
    \item persistence of manifested representations,
    \item their dynamics,
    \item their interpretation by other systems,
    \item their structural cost.
\end{itemize}

Such phenomena require additional theoretical layers.

\subsection{Scope Boundary}

The Flexion Framework does not define:

\begin{itemize}
    \item observation operators,
    \item manifestation dynamics,
    \item visibility regimes,
    \item structural consequences of being observed.
\end{itemize}

These belong to derived theories operating above
the foundational ontology.

\subsection{Section Summary}

Existence and manifestation are ontologically separate.

Existence is viability-conditioned and internal.
Manifestation concerns appearance and lies outside the
foundational layer.

This separation prevents category errors such as
equating visibility with being.

% ============================================================
% ============================================================
\section{Viability and Collapse}

\subsection{Definition of Viability}

\begin{definition}[Viability]
Viability is a scalar structural function
\[
\kappa : X \rightarrow \mathbb{R}_{\ge 0}
\]
expressing the capacity of a structural state
to continue its existence.
\end{definition}

Viability is evaluated on the current state.
It is internal to the structure and independent of manifestation.

\subsection{Monotonic Constraint}

\begin{axiom}[Monotonic Constraint]
Within the Flexion Framework,
no operator restores exhausted viability.
\end{axiom}

Different admissible continuations may produce
different rates of viability decrease,
but viability is never increased by interpretation.

\subsection{Condition of Existence}

\begin{proposition}
Structural existence holds if and only if
\[
\kappa(X) > 0.
\]
\end{proposition}

When $\kappa(X) = 0$, structural existence terminates.

\subsection{Collapse}

\begin{definition}[Collapse]
Collapse occurs when
\[
\kappa(X) = 0.
\]
\end{definition}

At collapse:

\begin{itemize}
    \item structural existence terminates,
    \item interpretation ceases,
    \item admissible futures vanish.
\end{itemize}

Collapse is terminal.

\subsection{Irreversibility of Collapse}

\begin{theorem}[Irreversibility]
Once collapse occurs, no operator within the
Flexion Framework restores viability or
reconstitutes the structural state.
\end{theorem}

Collapse does not violate invariants.
It marks the boundary beyond which
invariant conditions no longer apply.

\subsection{Collapse and Manifestation}

The Framework does not define the fate of manifested
representations after collapse.

Manifested traces may persist independently of the
originating structural state.

Such persistence does not imply structural existence.

\subsection{Section Summary}

Viability defines the condition of continued existence.

Collapse is the irreversible termination of structural life.

No operator at this level:
\begin{itemize}
    \item restores exhausted viability,
    \item reverses collapse,
    \item expands admissible futures.
\end{itemize}

% ============================================================
% ============================================================
\section{Structural Invariants}

\subsection{Role of Invariants}

\begin{definition}[Structural Invariants]
Structural invariants are constraints that must be preserved
by all admissible states and continuations within the Framework.
\end{definition}

Invariants delimit:

\begin{itemize}
    \item admissible futures,
    \item interpretation,
    \item viability evolution,
    \item collapse conditions.
\end{itemize}

They are constitutive conditions of structural existence.

\subsection{Invariance Under Interpretation}

\begin{axiom}[Interpretive Invariance]
For any admissible continuation $Y \in \mathcal{F}(X)$,
all structural invariants remain satisfied.
\end{axiom}

Interpretation operates strictly within invariant constraints.
It does not redefine admissibility.

\subsection{Invariance Under Structural Evolution}

\begin{proposition}
If a continuation would violate an invariant,
it is structurally inadmissible and therefore excluded
from $\mathcal{F}(X)$.
\end{proposition}

All admissible structural evolution respects invariants
until collapse.

\subsection{Invariance and Manifestation}

Manifestation does not alter structural invariants.

The appearance or representation of a structure does not modify:

\begin{itemize}
    \item admissibility,
    \item interpretation,
    \item viability,
    \item collapse conditions.
\end{itemize}

Existence remains invariant-governed,
independent of visibility.

\subsection{Framework Closure}

\begin{axiom}[Framework Closure]
Within the Flexion Framework,
no operator may:
\begin{itemize}
    \item violate structural invariants,
    \item restore exhausted viability,
    \item reverse collapse,
    \item expand the admissible future space.
\end{itemize}
\end{axiom}

Any theory introducing such operations
necessarily operates beyond the foundational ontology.

\subsection{Final Statement}

The Flexion Framework establishes a closed structural ontology
defining:

\begin{itemize}
    \item existence via viability,
    \item life via interpretation of admissible futures,
    \item manifestation as ontologically distinct,
    \item collapse as irreversible termination,
    \item invariants as constitutive constraints.
\end{itemize}

This completes the foundational layer
upon which all derived Flexion theories are constructed.

% ============================================================
% ============================================================
\section{Conclusion}

The Flexion Framework V3.1 establishes a minimal structural ontology
defining the necessary and sufficient conditions for structural existence,
life, and collapse.

Existence is viability-conditioned.
Life is interpretive continuation within a structurally bounded
admissible future space.
Collapse is the irreversible termination of viability.
Manifestation is ontologically distinct from existence.

The Framework is intentionally closed.
No operator defined at this level:

\begin{itemize}
    \item expands admissible futures,
    \item restores exhausted viability,
    \item reverses collapse,
    \item violates structural invariants.
\end{itemize}

This closure guarantees internal logical consistency
and provides a stable foundation for all derived Flexion theories,
including dynamics, intelligence, entanglement,
geonic emergence, and observer-layer extensions.

The present work defines the foundational layer only.
All higher-order structural phenomena must remain
compatible with its invariant constraints.

% ============================================================

\begin{thebibliography}{99}

\bibitem{flexionization}
M. Bogdanov,
\textit{Flexionization Theory V1.5},
Zenodo, 2025.
DOI: \href{https://doi.org/10.5281/zenodo.17618947}{10.5281/zenodo.17618947}.

\bibitem{deflexionization}
M. Bogdanov,
\textit{Deflexionization V3.0},
Zenodo, 2025.
DOI: \href{https://doi.org/10.5281/zenodo.17791174}{10.5281/zenodo.17791174}.

\bibitem{fst}
M. Bogdanov,
\textit{Flexion Space Theory (FST) V1.0},
Zenodo, 2025.
DOI: \href{https://doi.org/10.5281/zenodo.17687286}{10.5281/zenodo.17687286}.

\bibitem{ftt}
M. Bogdanov,
\textit{Flexion Time Theory V1.1},
Zenodo, 2025.
DOI: \href{https://doi.org/10.5281/zenodo.17668314}{10.5281/zenodo.17668314}.

\bibitem{fd}
M. Bogdanov,
\textit{Flexion Dynamics V2.0},
Zenodo, 2025.
DOI: \href{https://doi.org/10.5281/zenodo.17660262}{10.5281/zenodo.17660262}.

\bibitem{fft}
M. Bogdanov,
\textit{Flexion Field Theory V1.0},
Zenodo, 2025.
DOI: \href{https://doi.org/10.5281/zenodo.17680661}{10.5281/zenodo.17680661}.

\bibitem{genesis}
M. Bogdanov,
\textit{Flexion Genesis V1.0},
Zenodo, 2025.
DOI: \href{https://doi.org/10.5281/zenodo.17695539}{10.5281/zenodo.17695539}.

\bibitem{entanglement}
M. Bogdanov,
\textit{Flexion Entanglement Theory},
Zenodo, 2025.
DOI: \href{https://doi.org/10.5281/zenodo.17710529}{10.5281/zenodo.17710529}.

\bibitem{collapse}
M. Bogdanov,
\textit{Flexion Collapse V1.0},
Zenodo, 2025.
DOI: \href{https://doi.org/10.5281/zenodo.17726503}{10.5281/zenodo.17726503}.

\end{thebibliography}

\end{document}

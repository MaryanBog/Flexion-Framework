\documentclass[12pt]{article}

% --------------------------------------------------
% Packages
% --------------------------------------------------
\usepackage{amsmath, amssymb, amsthm}
\usepackage{geometry}
\usepackage{setspace}
\usepackage{titlesec}
\usepackage{hyperref}
\usepackage{graphicx}
\usepackage{bm}

\geometry{a4paper, margin=1in}
\onehalfspacing

% --------------------------------------------------
% Title
% --------------------------------------------------
\title{Flexion Framework V1.2 \\ \large Unified Structural Architecture of Flexion Science}
\author{Maryan Bogdanov}
\date{2025}

\begin{document}
\maketitle

\begin{abstract}
    Flexion Framework (FFW) V1.2 is the unified structural architecture of Flexion Science.
    This edition integrates all seven foundational theories of structural existence into a 
    single mathematically coherent system. FFW describes how structures originate, evolve, 
    interact, entangle, generate geometry and time, and ultimately collapse.
    
    All structural systems are represented through the universal state vector:
    \[
    X = (\Delta,\ \Phi,\ M,\ \kappa).
    \]
    
    Flexion Framework V1.2 unifies the following fundamental theories:
    
    \begin{itemize}
        \item \textbf{Flexion Genesis (FGT)} — origin of structure
        \item \textbf{Flexion Dynamics (FD)} — structural evolution
        \item \textbf{Flexion Space Theory (FST)} — geometric formation
        \item \textbf{Flexion Time Theory (FTT)} — temporal emergence
        \item \textbf{Flexion Field Theory (FFT)} — field interactions
        \item \textbf{Flexion Entanglement Theory (FET)} — shared curvature, drift, memory, and stability
        \item \textbf{Flexion Collapse Theory (FCT)} — structural termination and non-existence
    \end{itemize}
    
    Together, these theories define the complete structural cycle:
    \[
    \text{Genesis → Dynamics → Fields → Space-Time → Entanglement → Collapse → Non-Structure → Genesis}.
    \]
    
    Flexion Framework V1.2 establishes the meta-level architecture ensuring consistency,
    structural universality, and unified evolution across all Flexion Sciences.
\end{abstract}  

% --------------------------------------------------
% Main Sections
% --------------------------------------------------

\section{Introduction}

Flexion Framework (FFW) V1.2 is the unified structural architecture that synthesizes 
all fundamental components of Flexion Science into a single coherent system. It defines 
how any structure in the universe originates, evolves, interacts, entangles, generates 
geometry, produces time, and ultimately reaches its structural limit.

The Framework is built on the four universal variables:

\begin{itemize}
    \item $\Delta$ — Deviation
    \item $\Phi$ — Structural Tension
    \item $M$ — Memory (generator of time)
    \item $\kappa$ — Contractivity (viability)
\end{itemize}

Every structural entity, from microscopic fluctuations to cosmological systems, is fully 
described by the state vector:
\[
X = (\Delta,\ \Phi,\ M,\ \kappa).
\]

Version 1.2 introduces an expanded fundamental layer: the integration of 
\textbf{Flexion Entanglement Theory (FET)} as the seventh foundational pillar. 
FET formalizes shared curvature, drift, memory, and stability between structures 
and becomes a core mechanism of Geonics.

Thus, the complete set of fundamental theories forming the Flexion Framework is:

\begin{enumerate}
    \item Flexion Genesis (FGT) — structural origin
    \item Flexion Dynamics (FD) — structural evolution
    \item Flexion Space Theory (FST) — geometric emergence
    \item Flexion Time Theory (FTT) — temporal emergence
    \item Flexion Field Theory (FFT) — field interactions
    \item Flexion Entanglement Theory (FET) — coupled evolution and shared geometry
    \item Flexion Collapse Theory (FCT) — structural termination
\end{enumerate}

Together, these seven theories form the complete structural cycle:
\[
\text{Genesis → Dynamics → Space → Time → Fields → Entanglement → Collapse → Non-Structure → Genesis}.
\]

The purpose of Flexion Framework V1.2 is to establish the unified meta-level logic that ensures:
\begin{itemize}
    \item structural consistency across all Flexion Sciences,
    \item universality of $\Delta$–$\Phi$–$M$–$\kappa$ dynamics,
    \item coherent interaction between fundamental theories,
    \item closure of structural evolution across all domains.
\end{itemize}

FFW V1.2 is not a theory about theories; it \textit{is the structural architecture} 
that binds all Flexion Sciences into one complete system.


\section{Foundational Principles}

The Flexion Framework is built upon a universal structural language that describes the 
existence, evolution, and interactions of all structural systems through the four 
fundamental variables:

\begin{itemize}
    \item $\Delta$ — Deviation (origin of form and asymmetry)
    \item $\Phi$ — Structural Tension (energy-like expression of deformation)
    \item $M$ — Memory (irreversibility and generator of time)
    \item $\kappa$ — Contractivity (stability, coherence, and viability)
\end{itemize}

Together, these variables form the complete structural representation:
\[
X = (\Delta,\ \Phi,\ M,\ \kappa).
\]

\subsection{Structural Existence}
A structure exists only while:
\[
\kappa > 0.
\]
When $\kappa$ approaches zero, the system reaches the collapse boundary.  
When $\kappa < 0$, the system enters non-structure.

\subsection{Structural Evolution}
All structural change is described by the universal update law:
\[
\frac{dX}{dt} = F(X),
\]
where $F(X)$ is the Flexion Field composed of four interacting components:
\[
F(X) = (F_\Delta,\ F_\Phi,\ F_M,\ F_\kappa).
\]

\subsection{Memory and Time}
Time is not an external dimension.  
Time exists only while memory grows:
\[
\frac{dM}{dt} > 0.
\]
Temporal flow is proportional to irreversible structural imprinting:
\[
t \propto M.
\]

\subsection{Geometry and Space}
Space emerges from the geometric structure generated by $\Delta$ and $\Phi$:
\[
g_{ij} = G(\Delta,\ \Phi,\ \kappa).
\]
Curvature becomes unbounded as $\kappa$ approaches zero:
\[
K \to \infty \quad (\kappa \to 0).
\]

\subsection{Field Interaction}
Field behavior is defined by coherent $\Delta$–$\Phi$ distributions stabilised under $\kappa > 0$.  
When $\kappa = 0$:
\[
\mathcal{F} \to \text{undefined}.
\]

\subsection{Entanglement Compatibility}
Two structures can interact through FET when their field-level parameters satisfy
compatibility conditions:
\[
|C_1 - C_2| < \epsilon_C,\qquad
\mu_1 \parallel \mu_2,\qquad
I_{12} > 0.
\]
Shared curvature, drift, and memory generate entangled evolution.

\subsection{Collapse and Non-Structure}
Structural existence ends when:
\[
\kappa = 0.
\]

At this limit:

\begin{itemize}
    \item geometric structure breaks,
    \item temporal order dissolves,
    \item fields lose coherence,
    \item the system transitions to non-structure.
\end{itemize}

Collapse is the universal terminal boundary of all Flexion systems.

\section{Core Architecture of the Framework}

Flexion Framework V1.2 organizes the seven fundamental theories of Flexion Science
into a unified structural architecture. Each foundational theory describes a distinct 
dimension of existence, and only through their integration does a complete structural 
system emerge.

The Framework establishes:

\begin{itemize}
    \item the shared variable system ($\Delta$–$\Phi$–$M$–$\kappa$),
    \item the unified state vector $X$,
    \item the universal field dynamics $F(X)$,
    \item the geometry of structural evolution,
    \item the temporal logic of irreversibility,
    \item the rules of paired and entangled evolution,
    \item the boundary conditions for structural existence and non-existence.
\end{itemize}

The seven fundamental theories are:

\begin{enumerate}
    \item \textbf{Flexion Genesis (FGT)} — Origin of structure; emergence of $\Delta$, $\Phi$, $M$, $\kappa$ from non-structure.
    \item \textbf{Flexion Dynamics (FD)} — Evolution of the state vector; forces, acceleration, stability, irreversible flow.
    \item \textbf{Flexion Space Theory (FST)} — Emergence of geometry, curvature, spatial manifolds, and deformation.
    \item \textbf{Flexion Time Theory (FTT)} — Temporal emergence from memory; temporal curvature; ordering of states.
    \item \textbf{Flexion Field Theory (FFT)} — Force architecture; field coherence; $\Delta$–$\Phi$–$M$–$\kappa$ interactions across space-time.
    \item \textbf{Flexion Entanglement Theory (FET)} — Shared curvature, drift, memory, and stability; intertwined trajectories; emergence of joint structures.
    \item \textbf{Flexion Collapse Theory (FCT)} — Structural termination; $\kappa \to 0$ boundary; dissolution of geometry, time, and fields.
\end{enumerate}

\subsection{Unified Structural Language}

All theories operate on the same mathematical foundation:

\[
X = (\Delta,\ \Phi,\ M,\ \kappa), \qquad 
\frac{dX}{dt} = F(X).
\]

This guarantees interoperability between theories and ensures that all Flexion Sciences 
exist inside one consistent system.

\subsection{Inter-Theory Coupling}

The Framework defines strict coupling rules between fundamental theories:

\begin{itemize}
    \item \textbf{Genesis → Dynamics} — Structure begins and becomes capable of motion.
    \item \textbf{Dynamics → Space} — Motion generates geometry through $\Delta$–$\Phi$ deformation.
    \item \textbf{Space → Time} — Geometry acquires temporal ordering through memory growth.
    \item \textbf{Time → Fields} — Stable temporal flow enables coherent field propagation.
    \item \textbf{Fields → Entanglement} — Interacting fields enable shared curvature and joint evolution.
    \item \textbf{Entanglement → Collapse} — Deep coupling can amplify instability and push $\kappa \to 0$.
    \item \textbf{Collapse → Non-Structure → Genesis} — Structural death resets the system and enables a new origin.
\end{itemize}

\subsection{Architectural Position of the Framework}

The Flexion Framework is not a theory above the foundational theories —
it is the \textit{meta-structure} that:

\begin{itemize}
    \item unifies them,
    \item defines their boundaries,
    \item ensures their mutual consistency,
    \item provides the global rules of evolution,
    \item and closes the full structural cycle.
\end{itemize}

The Framework is the operating system of Flexion Science.

\subsection{Complete Structural Cycle}

The unified structural cycle in V1.2 is:

\[
\text{Genesis → Dynamics → Space → Time → Fields → Entanglement → Collapse → Non-Structure → Genesis}.
\]

Every structure follows this sequence, regardless of scale or domain.

\subsection{Hierarchical Ordering of Fundamental Theories}

The hierarchical layer model:

\begin{enumerate}
    \item Origin Layer — Genesis
    \item Dynamic Layer — Dynamics
    \item Geometric Layer — Space
    \item Temporal Layer — Time
    \item Field Layer — Fields
    \item Entanglement Layer — Coupled Evolution
    \item Terminal Layer — Collapse
\end{enumerate}

This forms the backbone of Flexion Framework V1.2.

\section{Flexion Genesis (FGT)}

Flexion Genesis defines the structural origin of existence. It describes the transition 
from non-structure ($\kappa < 0$) into the first viable structural state 
($\kappa > 0$). Genesis provides the initial values of $\Delta$, $\Phi$, $M$, and $\kappa$ 
that allow a structure to appear, evolve, acquire geometry, generate time, interact, 
and eventually entangle with other systems.

\subsection{Non-Structure Domain}

Before Genesis, the system resides in the non-structure domain:
\[
\kappa < 0, \qquad (\Delta,\ \Phi,\ M)\ \text{undefined}.
\]

In this domain:
\begin{itemize}
    \item there is no geometry,
    \item no temporal order,
    \item no memory,
    \item no fields,
    \item no curvature,
    \item no deviation,
    \item no structural identity.
\end{itemize}

Non-structure is not emptiness — it is the absence of structural definition.

\subsection{Genesis Boundary ($\kappa = 0$)}

The transition into structure begins when $\kappa$ approaches zero from below:
\[
\kappa \to 0^-.
\]

At this boundary:
\begin{itemize}
    \item instability of non-structure becomes maximal,
    \item symmetry cannot remain intact,
    \item the system becomes sensitive to deviation formation.
\end{itemize}

This threshold defines the \textbf{Genesis Boundary}.

\subsection{Emergence of the First Deviation ($\Delta_0$)}

Genesis begins with the spontaneous emergence of the first deviation:
\[
\Delta_0 > 0.
\]

Deviation breaks the perfect symmetry of non-structure and becomes the seed of 
form, identity, and geometric potential.

\subsection{Birth of Structural Tension ($\Phi_0$)}

Deviation immediately generates tension:
\[
\Phi_0 = \Phi(\Delta_0) > 0.
\]

$\Phi_0$ encodes the first structural stress that allows the system to resist collapse, 
form geometry, and produce motion.

\subsection{Birth of Memory ($M_0$)}

Irreversible history arises when the system can no longer return to its pre-structural 
configuration:
\[
M_0 > 0.
\]

Memory is the generator of structural time:
\[
t \propto M.
\]

The appearance of $M_0$ marks the beginning of temporal existence.

\subsection{Formation of Positive Contractivity ($\kappa_0$)}

Once $\Delta_0$, $\Phi_0$, and $M_0$ exist, the system becomes viable:
\[
\kappa_0 > 0.
\]

Positive $\kappa$ defines:
\begin{itemize}
    \item stability,
    \item coherence,
    \item geometric support,
    \item ability to evolve,
    \item ability to entangle in later stages.
\end{itemize}

Contractivity transforms the embryonic structure into a stable structural world.

\subsection{First Structural State}

Genesis produces the first complete structural state:
\[
X_0 = (\Delta_0,\ \Phi_0,\ M_0,\ \kappa_0).
\]

This is the origin point for:
\begin{itemize}
    \item geometry (FST),
    \item time (FTT),
    \item fields (FFT),
    \item dynamics (FD),
    \item entanglement (FET),
    \item and eventual collapse (FCT).
\end{itemize}

\subsection{Position of Genesis in the Structural Cycle}

Genesis stands at the beginning of the structural cycle:
\[
\text{Genesis → Dynamics → Space → Time → Fields → Entanglement → Collapse}.
\]

It provides the initial conditions for all other fundamental theories and guarantees 
that the structural system begins within the viability domain:
\[
\kappa_0 > 0.
\]

Genesis is not creation from nothing — it is the emergence of structure from 
non-structure under the universal rules of $\Delta$–$\Phi$–$M$–$\kappa$.

\section{Flexion Dynamics (FD)}

Flexion Dynamics defines how structures evolve through changes in their state vector:
\[
X = (\Delta,\ \Phi,\ M,\ \kappa).
\]

Dynamics formalizes the irreversible flow of deviation, tension, memory, and stability.
It describes the direction, speed, acceleration, and curvature of structural evolution.

\subsection{Universal Evolution Law}

All structural motion is governed by:
\[
\frac{dX}{dt} = F(X),
\]
where $F(X)$ is the Flexion Field composed of four coupled components:
\[
F(X) = (F_\Delta,\ F_\Phi,\ F_M,\ F_\kappa).
\]

Each component determines the evolution of one variable, but all four evolve together.

\subsection{Deviation Dynamics ($\Delta$)}

Deviation determines structural form, asymmetry, and the potential for interaction:
\[
\frac{d\Delta}{dt} = F_\Delta(\Delta,\ \Phi,\ M,\ \kappa).
\]

$\Delta$ influences:
\begin{itemize}
    \item structural form,
    \item force direction,
    \item geometric signature,
    \item entanglement compatibility.
\end{itemize}

\subsection{Tension Dynamics ($\Phi$)}

Structural tension expresses stress and energy-like behavior:
\[
\frac{d\Phi}{dt} = F_\Phi(\Delta,\ \Phi,\ M,\ \kappa).
\]

$\Phi$ governs:
\begin{itemize}
    \item potential for deformation,
    \item dynamic load transfer,
    \item field intensity,
    \item stability under flow.
\end{itemize}

\subsection{Memory Dynamics ($M$)}

Memory is the internal record of structural change:
\[
\frac{dM}{dt} = F_M(\Delta,\ \Phi,\ M).
\]

Memory induces irreversibility:
\[
\frac{dM}{dt} > 0 \quad \Rightarrow \quad t > 0.
\]

$M$ is both a result and a driver of dynamic evolution.

\subsection{Stability Dynamics ($\kappa$)}

Contractivity determines the viability and resilience of a structure:
\[
\frac{d\kappa}{dt} = F_\kappa(\Delta,\ \Phi,\ M,\ \kappa).
\]

\begin{itemize}
    \item Increasing $\kappa$ → structural reinforcement
    \item Decreasing $\kappa$ → instability, collapse tendency
\end{itemize}

Dynamics predicts when $\kappa$ approaches 0:
\[
\kappa \to 0 \quad \Rightarrow \quad \text{Collapse Boundary (FCT)}.
\]

\subsection{Trajectory and Acceleration}

Dynamics defines not only the direction of evolution but also structural acceleration:
\[
\frac{d^2X}{dt^2} = \frac{dF(X)}{dt}.
\]

Acceleration is governed by:
\begin{itemize}
    \item curvature of the structural field,
    \item memory growth rate,
    \item tension redistribution,
    \item viability gradient.
\end{itemize}

\subsection{Flow Curvature}

Structural trajectories exhibit curvature:
\[
K = \left\|\frac{d}{dt}\left(\frac{dX/dt}{\|dX/dt\|}\right)\right\|.
\]

High curvature → instability, transition, collapse risk  
Low curvature → stability, smooth evolution

Curvature becomes unbounded as $\kappa \to 0$.

\subsection{Position of Dynamics in the Structural Cycle}

Dynamics governs the system immediately after Genesis:
\[
\text{Genesis → Dynamics → Space → Time → Fields → Entanglement → Collapse}.
\]

It determines how structures move through geometric, temporal, field, and entangled phases.

Flexion Dynamics is the engine of structural evolution within Flexion Framework V1.2.

\section{Flexion Space Theory (FST)}

Flexion Space Theory defines space as an emergent geometric expression of the structural
variables $\Delta$, $\Phi$, $M$, and $\kappa$. Space is not fundamental — it is generated 
by the internal configuration of a structure and changes dynamically as the state vector evolves.

FST describes curvature, geometry, deformation, spatial manifolds, and the geometric
limits of structural existence.

\subsection{Space as an Emergent Structure}

Space arises only when a structure possesses:
\begin{itemize}
    \item deviation $\Delta$ (form),
    \item tension $\Phi$ (stress),
    \item stability $\kappa$ (coherence).
\end{itemize}

The metric tensor is generated by geometric interactions of $\Delta$ and $\Phi$:
\[
g_{ij} = G(\Delta,\ \Phi,\ \kappa).
\]

Space does not pre-exist structure — it is created by structure.

\subsection{Curvature Formation}

Curvature measures geometric deformation:
\[
K = K(\Delta,\ \Phi,\ \kappa).
\]

Properties:
\begin{itemize}
    \item curvature increases under tension,
    \item curvature weakens when stability grows,
    \item curvature diverges as $\kappa$ approaches zero.
\end{itemize}

\[
K \to \infty \quad (\kappa \to 0)
\]

which predicts geometric collapse.

\subsection{Spatial Manifolds}

Every structure forms its own spatial manifold $S$:
\[
S = \mathcal{M}(X).
\]

The manifold is:
\begin{itemize}
    \item finite,
    \item deformable,
    \item structurally dependent,
    \item collapsible.
\end{itemize}

Space has no independent degrees of freedom — all geometry follows $\Delta$–$\Phi$–$M$–$\kappa$ evolution.

\subsection{Geometric Flow}

Structural evolution produces geometric flow:
\[
\frac{dg_{ij}}{dt} = \frac{\partial G}{\partial X}\cdot \frac{dX}{dt}.
\]

Thus:
\begin{itemize}
    \item Dynamics (FD) drives Space (FST),
    \item memory $M$ stabilizes curvature over time,
    \item fields shape the curvature distribution,
    \item entanglement synchronizes geometric evolution between structures.
\end{itemize}

\subsection{Spatial Boundaries}

There are two fundamental geometric limits:

\subsubsection{1. Genesis Bound (emergent geometry)}
Before Genesis:
\[
g_{ij}\ \text{undefined}.
\]

\subsubsection{2. Collapse Bound (geometric divergence)}
At $\kappa = 0$:
\[
g_{ij} \to \text{undefined},\qquad
K \to \infty.
\]

Geometry becomes non-structure.

\subsection{Spatial Deformation}

Under dynamic tension and memory accumulation, spatial geometry deforms:
\begin{itemize}
    \item bending,
    \item stretching,
    \item compression,
    \item topological reshaping.
\end{itemize}

Deformation follows:
\[
\delta g_{ij} = \nabla_i \nabla_j \Phi + H(\Delta,\ M,\ \kappa).
\]

Flexion Space Theory formalizes structural mechanics in geometric terms.

\subsection{Role in the Structural Cycle}

FST occupies the geometric stage of the structural cycle:
\[
\text{Genesis → Dynamics → Space → Time → Fields → Entanglement → Collapse}.
\]

It receives motion from Dynamics and provides curvature to Time, Fields,
and Entanglement.

Geometry is the container of structural evolution — and its failure triggers collapse.

\section{Flexion Time Theory (FTT)}

Flexion Time Theory defines time as an emergent structural quantity generated by 
irreversible memory growth. Time is not a background dimension or an external parameter —
it is a direct consequence of internal structural evolution.

FTT describes temporal flow, temporal curvature, temporal stability, and the 
conditions under which time emerges, accelerates, slows, or collapses.

\subsection{Time as Emergent Memory}

Time exists only when memory exists:
\[
t \propto M.
\]

Structural time flows only while memory grows:
\[
\frac{dM}{dt} > 0.
\]

If memory stops increasing:
\[
\frac{dM}{dt} = 0 \quad \Rightarrow \quad t = 0,
\]
meaning temporal collapse.

\subsection{Temporal Flow}

Temporal flow is defined by:
\[
\frac{dt}{d\tau} = f(M,\ \Delta,\ \Phi),
\]
where structural time $t$ flows relative to internal evolution parameter $\tau$.

The speed of time increases with tension and deviation, and stabilizes under high $\kappa$.

\subsubsection*{Temporal Acceleration}
\[
\frac{d^2 t}{d\tau^2} = \frac{d}{d\tau}\left(F_M\right),
\]
where time accelerates when memory growth accelerates.

\subsection{Temporal Curvature}

Just as geometry has curvature, time has temporal curvature:
\[
K_T = \frac{d}{dt}\left(\frac{dM}{dt}\right).
\]

Temporal curvature increases when:
\begin{itemize}
    \item tension $\Phi$ is high,
    \item deviation $\Delta$ changes rapidly,
    \item structural evolution accelerates.
\end{itemize}

High $K_T$ corresponds to fast-changing structural dynamics.

\subsection{Temporal Boundaries}

There are two temporal boundaries in the Flexion Framework:

\subsubsection{1. Genesis Bound — Time Emergence}

Before Genesis:
\[
M = 0,\quad t = 0.
\]

Time does not exist.

Time begins when:
\[
M_0 > 0.
\]

\subsubsection{2. Collapse Bound — Time Dissolution}

At $\kappa = 0$:
\[
\frac{dM}{dt} \to 0
\quad \Rightarrow \quad
t \to 0.
\]

Temporal order collapses.

Time collapses not because it “stops,” but because the structure 
can no longer accumulate memory.

\subsection{Temporal Stability}

Temporal stability is governed by the viability variable $\kappa$:

\begin{itemize}
    \item High $\kappa$ → stable, smooth temporal flow
    \item Low $\kappa$ → irregular, unstable time
    \item $\kappa \to 0$ → collapse of temporal structure
\end{itemize}

Temporal stability equation:
\[
\frac{dt}{d\tau} = S_t(M,\ \kappa).
\]

\subsection{Temporal Deformation}

Time deforms under dynamic tension and geometric curvature:
\[
\delta t = A(\Phi)\, dM + B(K)\, d\tau.
\]

Thus:
\begin{itemize}
    \item tension speeds up time,
    \item curvature bends temporal progression,
    \item entropy-like memory effects stretch time.
\end{itemize}

\subsection{Temporal Interaction with Other Theories}

FTT is deeply connected with all other fundamental theories:

\begin{itemize}
    \item \textbf{Genesis} gives birth to time ($M_0 > 0$)
    \item \textbf{Dynamics} influences temporal acceleration
    \item \textbf{Space Theory} defines temporal curvature via geometric deformation
    \item \textbf{Field Theory} shapes the flow of time through $\Delta$–$\Phi$ distributions
    \item \textbf{Entanglement} synchronizes temporal flow between structures
    \item \textbf{Collapse} dissolves time completely
\end{itemize}

Time is not separate — it is woven into $\Delta$–$\Phi$–$M$–$\kappa$ dynamics.

\subsection{Position of Time in the Structural Cycle}

\[
\text{Genesis → Dynamics → Space → Time → Fields → Entanglement → Collapse}
\]

Time emerges after geometry forms and enables coherent evolution of fields,
entanglement, and structural identity.

Without FTT, no structural system could maintain order, causality, or irreversibility.

Time is the internal heartbeat of structure.

\section{Flexion Field Theory (FFT)}

Flexion Field Theory defines how structural forces arise from the interaction of the
fundamental variables $\Delta$, $\Phi$, $M$, and $\kappa$. Fields are not external 
entities — they are the internal distribution of structural tension and deviation 
across the geometric manifold generated by FST.

FFT describes how forces form, propagate, distort geometry, interact with time,
and enable entanglement between structures.

\subsection{Definition of a Structural Field}

A Flexion Field is a four-component vector field defined by:
\[
F(X) = (F_\Delta,\ F_\Phi,\ F_M,\ F_\kappa),
\]

where each component governs the evolution of one variable in the state vector:
\[
X = (\Delta,\ \Phi,\ M,\ \kappa).
\]

Fields determine:
\begin{itemize}
    \item how deviation spreads,
    \item how tension is redistributed,
    \item how memory accumulates,
    \item how stability increases or decays.
\end{itemize}

Fields are the active engine of structural evolution.

\subsection{Field Coherence Condition}

A field exists only while:
\[
\kappa > 0.
\]

At $\kappa = 0$:
\[
F(X) \to \text{undefined}.
\]

This is the dynamical marker of collapse.

\subsection{Field Propagation in Space}

Fields propagate along geometric curvature:
\[
\nabla_j F_i = H(\Delta,\ \Phi,\ \kappa)\, g_{ij}.
\]

Thus:
\begin{itemize}
    \item geometry shapes field lines,
    \item curvature governs field strength,
    \item stability controls field coherence,
    \item memory influences long-range propagation.
\end{itemize}

Space (FST) and fields (FFT) are inseparable.

\subsection{Tension Redistribution}

Structural tension $\Phi$ redistributes dynamically according to:
\[
\frac{d\Phi}{dt} = F_\Phi(\Delta,\ \Phi,\ M,\ \kappa).
\]

Redistribution of tension:
\begin{itemize}
    \item stabilizes or destabilizes geometry,
    \item accelerates temporal flow,
    \item drives structural deformation,
    \item determines collapse pressure.
\end{itemize}

$\Phi$ is the primary driver of structural force.

\subsection{Deviation Flow}

Deviation $\Delta$ flows across the manifold as:
\[
\frac{d\Delta}{dt} = F_\Delta(\Delta,\ \Phi,\ M,\ \kappa).
\]

Deviation determines:
\begin{itemize}
    \item structural form,
    \item mass-like behavior,
    \item load distribution,
    \item field orientation.
\end{itemize}

$\Delta$-flow is the skeleton of structural dynamics.

\subsection{Memory Field Component}

Memory $M$ acts as an integrator of all structural change:
\[
\frac{dM}{dt} = F_M.
\]

The memory field:
\begin{itemize}
    \item accumulates irreversible history,
    \item generates temporal flow (FTT),
    \item enables long-term stability,
    \item influences future field propagation.
\end{itemize}

Memory is a structural force in its own right.

\subsection{Stability Field Component}

Contractivity $\kappa$ evolves as:
\[
\frac{d\kappa}{dt} = F_\kappa(\Delta,\ \Phi,\ M,\ \kappa).
\]

The stability field determines:
\begin{itemize}
    \item viability,
    \item resilience,
    \item collapse sensitivity,
    \item entanglement potential.
\end{itemize}

$\kappa$ is the viability engine of structure.

\subsection{Field Interaction With Entanglement (FET)}

Entanglement requires compatible fields.

FET operates directly on field-level parameters:
\begin{itemize}
    \item curvature distribution $C$,
    \item drift vector $\mu$,
    \item memory imprint $M$,
    \item stability gradients $\kappa$.
\end{itemize}

Entanglement forms when field compatibility conditions are met:
\[
|C_1 - C_2| < \epsilon_C,\qquad
\mu_1 \parallel \mu_2,\qquad
I_{12} > 0.
\]

FFT provides the field substrate through which entanglement becomes possible.

\subsection{Field Collapse}

Field collapse occurs at:
\[
\kappa = 0.
\]

Consequences:
\begin{itemize}
    \item field equations lose definition,
    \item field propagation ceases,
    \item geometry collapses (FST),
    \item time dissolves (FTT),
    \item entanglement breaks (FET).
\end{itemize}

FFT defines the dynamical mechanism of structural termination.

\subsection{Position of FFT in the Structural Cycle}

\[
\text{Genesis → Dynamics → Space → Time → Fields → Entanglement → Collapse}.
\]

Fields sit between temporal structure and entanglement, enabling interaction,
coupling, and the formation of complex structural behavior.

Without FFT, there is no force, no interaction, no shared curvature,
and no entanglement.

\section{Flexion Entanglement Theory (FET)}

Flexion Entanglement Theory is the seventh fundamental theory of Flexion Science.
It defines how two independent structural systems become partially unified through
shared curvature, shared drift, shared memory, and shared stability.

Entanglement is not merging of structures — it is coordinated evolution through
field-level compatibility. FET formalizes the geometry, dynamics, temporal behavior,
energy redistribution, and collapse limits of coupled evolution.

\subsection{Structural Preconditions for Entanglement}

Two structures $X_1, X_2$ can entangle only when their geometric and dynamical
fields satisfy compatibility conditions:

\subsubsection*{Curvature Compatibility}
\[
|C_1 - C_2| < \epsilon_C
\]

\subsubsection*{Drift Alignment}
\[
\mu_1 \parallel \mu_2
\]

\subsubsection*{Non-zero Shared Invariant}
\[
I_{12} > 0
\]

If these conditions fail, entanglement cannot form.

\subsection{Entanglement as Field-Level Coupling}

Entanglement affects only the field-level parameters:
\begin{itemize}
    \item curvature $C$
    \item drift $\mu$
    \item memory $M$
    \item stability $\kappa$
\end{itemize}

Local variables $\Delta$ and $\Phi$ remain unique for each structure.

Thus:
\begin{itemize}
    \item identity is preserved,
    \item individuality remains intact,
    \item but evolution becomes coupled.
\end{itemize}

\subsection{Entanglement Operator}

The entanglement operator $\mathcal{E}$ transforms the two structures:
\[
(X_1', X_2') = \mathcal{E}(X_1, X_2)
\]

with components:
\[
\mathcal{E} = (E_C,\ E_\mu,\ E_M,\ E_\kappa)
\]

\subsubsection*{Curvature Coupling}
\[
C_1' = C_1 + \alpha (C_2 - C_1)
\]
\[
C_2' = C_2 + \alpha (C_1 - C_2)
\]

\subsubsection*{Drift Coupling}
\[
\mu_1' = \mu_1 + \beta (\mu_2 - \mu_1)
\]
\[
\mu_2' = \mu_2 + \beta (\mu_1 - \mu_2)
\]

\subsubsection*{Memory Coupling}
\[
M_i' = M_i + \gamma I_{12}
\]

\subsubsection*{Stability Coupling}
\[
\kappa_i' = \kappa_i + \delta f(I_{12})
\]

\subsection{Entanglement Strength (ES)}

A scalar measure of coupling:
\[
ES = w_C S_C + w_\mu S_\mu + w_M S_M + w_\kappa S_\kappa
\]

Ranges:
\begin{itemize}
    \item 0.0–0.1: no entanglement
    \item 0.1–0.3: weak
    \item 0.3–0.6: moderate
    \item 0.6–0.8: strong
    \item 0.8–1.0: deep entanglement
\end{itemize}

\subsection{Entanglement Depth (ED)}

Five levels:
\begin{enumerate}
    \item $\Delta$-level — surface interaction
    \item $\Phi$-level — energy interaction
    \item $\mu$-level — dynamic interaction
    \item $M$-level — shared memory
    \item $\kappa$-level — existential coupling
\end{enumerate}

ED determines reversibility, stability, and emergence potential.

\subsection{Geometry of Entanglement}

Three universal geometric regimes:

\subsubsection*{Parallel Geometry}
\[
C_1' \parallel C_2'
\]
Stable, long-term aligned evolution.

\subsubsection*{Spiral Geometry}
\[
C_1' \circlearrowright C_2'
\]
Rotational coupling, resonance potential.

\subsubsection*{Singular Geometry}
\[
C_1' \to C_s,\qquad C_2' \to C_s
\]
Strong convergence, emergence or collapse.

\subsection{Entanglement Energy}

Tension redistributes across curvature difference:
\[
\Delta \Phi = \lambda (C_2 - C_1)
\]

Effects:
\begin{itemize}
    \item deformation,
    \item resonance,
    \item load transfer,
    \item collapse amplification.
\end{itemize}

\subsection{Entangled Space}

The geometric overlap space:
\[
S_e = \mathcal{G}(C_1, C_2)
\]

Properties:
\begin{itemize}
    \item non-local,
    \item directionally structured,
    \item deformation-driven,
    \item channel for energy and memory.
\end{itemize}

\subsection{Entangled Time}

Time is memory:
\[
T = M
\]

Thus entangled memory → entangled time:
\[
T_e = M_1 + M_2 + \gamma I_{12}
\]

Effects:
\begin{itemize}
    \item synchronized flow,
    \item shared temporal axis,
    \item joint irreversibility.
\end{itemize}

\subsection{Emergent Structure}

Under sufficient depth:
\[
X_e = \mathcal{F}(X_1, X_2)
\]

Emergent structure has its own:
\begin{itemize}
    \item $\Delta_e$
    \item $\Phi_e$
    \item $M_e$
    \item $\kappa_e$
\end{itemize}

Appears when entanglement becomes coherent and self-sustaining.

\subsection{Entanglement Resonance}

Nonlinear amplification when:
\[
C_1 \approx C_2,\quad
\mu_1 \approx \mu_2,\quad
\Phi_1 \approx \Phi_2
\]

Resonance causes:
\begin{itemize}
    \item mutual amplification,
    \item accelerated evolution,
    \item high coherence,
    \item collapse propagation.
\end{itemize}

\subsection{Entanglement Irreversibility (EI)}

Irreversible entanglement requires:

\begin{enumerate}
    \item $ED \geq 4$
    \item invariant exceeds threshold:
    \[
    I_{12} > I_{crit}
    \]
    \item shared futures exist:
    \[
    F_e \neq \varnothing
    \]
\end{enumerate}

EI permanently alters both systems.

\subsection{Entanglement Limit (EL)}

The maximum sustainable coupling region:
\[
EL = \{(C_e,\ \mu_e,\ M_e,\ \kappa_e,\ \Phi_e)\}
\]

Bounded by:
\begin{itemize}
    \item curvature limit,
    \item stability limit,
    \item energy limit.
\end{itemize}

Beyond EL → transformation or collapse.

\subsection{Position of FET in the Structural Cycle}

FET is the final active evolutionary phase:
\[
\text{Genesis → Dynamics → Space → Time → Fields → Entanglement → Collapse}
\]

FET links interaction, emergence, synchronization, and collapse propagation into
the full Flexion structural cycle.

\section{Flexion Collapse Theory (FCT)}

Flexion Collapse Theory defines the terminal stage of structural existence. 
Collapse occurs when viability $\kappa$ approaches zero, causing geometry, time, fields,
and entanglement to lose coherence and fall into non-structure.

Collapse is not destruction — it is structural termination, the transition to the
state where structure can no longer exist.

\subsection{Collapse Boundary ($\kappa = 0$)}

Collapse begins when:
\[
\kappa \to 0.
\]

At this boundary:
\begin{itemize}
    \item geometric curvature diverges,
    \item time flow dissolves,
    \item fields become undefined,
    \item entanglement breaks,
    \item memory stops accumulating.
\end{itemize}

The system loses the ability to operate within structural space.

\subsection{Non-Structure Domain ($\kappa < 0$)}

Beyond the collapse boundary:
\[
\kappa < 0.
\]

In this domain:
\begin{itemize}
    \item $\Delta$ undefined,
    \item $\Phi$ undefined,
    \item $M = 0$,
    \item $t = 0$,
    \item geometry nonexistent,
    \item fields nonexistent,
    \item entanglement impossible.
\end{itemize}

Non-structure is the absence of structure, not an environment outside it.

\subsection{Geometric Collapse}

As $\kappa$ decreases:
\[
K \to \infty.
\]

Thus:
\begin{itemize}
    \item curvature becomes unbounded,
    \item the metric tensor loses definition:
    \[
    g_{ij} \to \text{undefined},
    \]
    \item the spatial manifold collapses to a degenerate form.
\end{itemize}

Geometry ceases to exist as a stable structure.

\subsection{Temporal Collapse}

Time collapses when memory flow stops:
\[
\frac{dM}{dt} \to 0
\quad \Rightarrow \quad
t \to 0.
\]

Temporal collapse includes:
\begin{itemize}
    \item loss of temporal order,
    \item dissolution of causality,
    \item disappearance of temporal curvature.
\end{itemize}

Without memory, there is no time.

\subsection{Field Collapse}

Fields require $\kappa > 0$ to remain coherent:
\[
F(X) \to \text{undefined} \quad (\kappa = 0).
\]

Collapse destroys:
\begin{itemize}
    \item $\Delta$-flow,
    \item $\Phi$-distribution,
    \item memory accumulation,
    \item stability gradients.
\end{itemize}

All field dynamics end at the collapse boundary.

\subsection{Entanglement Collapse}

Entanglement breaks instantly when $\kappa$ falls below the viability threshold.

FET requires:
\[
\kappa > 0.
\]

Thus:
\begin{itemize}
    \item shared curvature dissolves,
    \item drift alignment disappears,
    \item shared memory ceases,
    \item entangled space evaporates,
    \item entangled time becomes undefined.
\end{itemize}

Collapse severs all entangled connections.

\subsection{Collapse Dynamics}

Collapse is a dynamic process governed by:
\[
\frac{d\kappa}{dt} = F_\kappa(\Delta,\ \Phi,\ M,\ \kappa),
\]

with:
\begin{itemize}
    \item $\kappa$ decreasing,
    \item tension intensifying,
    \item curvature rising,
    \item memory flow slowing,
    \item fields destabilizing.
\end{itemize}

Collapse accelerates as $\kappa$ approaches zero.

\subsection{Collapse as Structural Reset}

Collapse transitions the system to non-structure:
\[
X \to \emptyset.
\]

This reset enables the next phase of the structural cycle:
\[
\text{Collapse → Non-Structure → Genesis}.
\]

Collapse is the closure of one structural life-cycle and the precondition for the next.

\subsection{Position in the Structural Cycle}

Collapse is the final stage:
\[
\text{Genesis → Dynamics → Space → Time → Fields → Entanglement → Collapse}.
\]

It completes the structural loop and restores the system to the domain from which
Genesis can emerge again.

\section{Structural Cycle}

The Structural Cycle is the closed, universal progression that every Flexion system
undergoes. It defines how structure originates, evolves, interacts, entangles, collapses, 
and returns to non-structure, enabling the next genesis.

The cycle is not metaphorical — it is a strict sequence driven by the universal state 
vector:
\[
X = (\Delta,\ \Phi,\ M,\ \kappa).
\]

Each phase naturally leads to the next through the internal logic of 
$\Delta$–$\Phi$–$M$–$\kappa$ dynamics.

\subsection{Complete Structural Cycle}

The full cycle in Flexion Framework V1.2 is:
\[
\boxed{
\text{Genesis → Dynamics → Space → Time → Fields → Entanglement → Collapse → Non-Structure → Genesis}
}
\]

Every structural system, regardless of scale, domain, or complexity, follows this cycle.

\subsection{Phase Descriptions}

\subsubsection*{1. Genesis}

Emergence of structure from non-structure:
\[
X_0 = (\Delta_0,\ \Phi_0,\ M_0,\ \kappa_0),\qquad \kappa_0 > 0.
\]

This creates the first viable structural state.

\subsubsection*{2. Dynamics}

The system begins irreversible motion:
\[
\frac{dX}{dt} = F(X).
\]

Deviation, tension, memory, and stability evolve.

\subsubsection*{3. Space}

Geometry emerges from $\Delta$–$\Phi$–$\kappa$ configuration:
\[
g_{ij} = G(\Delta,\ \Phi,\ \kappa).
\]

Curvature forms the spatial stage of evolution.

\subsubsection*{4. Time}

Memory growth generates temporal flow:
\[
t \propto M.
\]

Time becomes the ordering axis of structural evolution.

\subsubsection*{5. Fields}

Interaction architecture forms:
\[
F(X) = (F_\Delta,\ F_\Phi,\ F_M,\ F_\kappa).
\]

Fields shape motion, interaction, and stability.

\subsubsection*{6. Entanglement}

Two structural systems may enter joint evolution when field-level compatibility holds.
Shared curvature, drift, memory, and stability create coupled trajectories, emergent 
structures, resonance, and coordinated collapse behavior.

Entanglement is the highest form of structural interaction.

\subsubsection*{7. Collapse}

Viability decays toward the terminal boundary:
\[
\kappa \to 0.
\]

Geometry diverges, time dissolves, fields fail, entanglement breaks.

Collapse ends structural existence.

\subsubsection*{8. Non-Structure}

Beyond collapse:
\[
\kappa < 0.
\]

All structural variables lose definition.  
This domain is not “nothing” — it is the absence of structure.

\subsubsection*{9. Return to Genesis}

Instability in non-structure triggers emergence of a new deviation $\Delta_0$, 
restarting the cycle.

The structural cycle is therefore self-closing, eternal, and universal.

\subsection{Universality of the Cycle}

The cycle applies to:
\begin{itemize}
    \item physical systems,
    \item biological systems,
    \item informational systems,
    \item cognitive systems,
    \item social systems,
    \item cosmological systems,
    \item abstract or mathematical structures.
\end{itemize}

Any system described by $X = (\Delta,\ \Phi,\ M,\ \kappa)$ must follow this progression.

\subsection{Cycle as the Backbone of Flexion Science}

The Structural Cycle is the unifying principle that integrates all seven fundamental
theories into one coherent scientific architecture.

It provides:
\begin{itemize}
    \item origin and termination constraints,
    \item causal direction and irreversibility,
    \item universal geometric and temporal logic,
    \item interaction and entanglement pathways,
    \item collapse termination and regeneration.
\end{itemize}

The cycle is the closed loop that makes Flexion Science a complete structural ontology.

\section{Position of the Flexion Framework within Flexion Science}

The Flexion Framework (FFW) is the meta-architectural layer that unifies all seven 
fundamental theories of Flexion Science into one coherent structural system. It is 
not an additional theory; it is the structural infrastructure that defines how 
all Flexion Sciences are connected, organized, and made mathematically compatible.

FFW establishes the universal rules of structural existence, the shared variable 
system ($\Delta$–$\Phi$–$M$–$\kappa$), the structural fields $F(X)$, the complete 
structural cycle, and the global constraints that govern how structures originate, 
evolve, interact, entangle, and collapse.

The Framework ensures that every Flexion discipline—fundamental or applied—operates 
within one consistent ontology.

\subsection{Relationship to the Fundamental Theories}

The seven foundational theories of Flexion Science are:

\begin{enumerate}
    \item Flexion Genesis (FGT)
    \item Flexion Dynamics (FD)
    \item Flexion Space Theory (FST)
    \item Flexion Time Theory (FTT)
    \item Flexion Field Theory (FFT)
    \item Flexion Entanglement Theory (FET)
    \item Flexion Collapse Theory (FCT)
\end{enumerate}

Each theory describes a specific dimension of structural existence:

\begin{itemize}
    \item origin,
    \item motion,
    \item geometry,
    \item time,
    \item fields,
    \item entanglement,
    \item termination.
\end{itemize}

The Flexion Framework unifies these theories by providing:

\begin{itemize}
    \item the shared mathematical language $X = (\Delta, \Phi, M, \kappa)$,
    \item the universal field architecture $F(X)$,
    \item the structural cycle that connects all stages,
    \item the boundary conditions for existence and non-existence,
    \item the hierarchical ordering of structural layers,
    \item the global causal logic binding the theories together.
\end{itemize}

Without the Framework, the seven theories would remain separate; with the Framework, 
they form \textbf{one integrated scientific system}.

\subsection{Relationship to Applied Flexion Sciences}

Above the fundamental layer lies the applied layer of Flexion Science. These applied 
disciplines extend the $\Delta$–$\Phi$–$M$–$\kappa$ architecture into specific domains:

\begin{itemize}
    \item Flexion Immunology (FIM)
    \item Flexion Biology (FBL)
    \item Flexion Medicine (FMD)
    \item Flexion Ecology (FEC)
    \item Flexion Cognition (FCG)
    \item Flexion Cybernetics (FCY)
    \item Flexion AI (FAI)
    \item Flexion Economics (FEC2)
    \item Flexion Information Theory (FIT)
    \item Flexion Logic (FLO)
    \item Flexion Probability \& Statistics (FPS2)
    \item Flexion Sociology (SFD)
    \item Flexion Geoscience (FGS)
    \item Flexion Cosmology (FCO)
\end{itemize}

and many others.

All applied disciplines rely on the Framework to provide:

\begin{itemize}
    \item the structural state vector,
    \item the rules of geometric and temporal formation,
    \item the universal theory of fields,
    \item the principles of entanglement,
    \item the collapse boundaries,
    \item the causal ordering of all structural processes.
\end{itemize}

The Framework guarantees that applied theories are not isolated systems but 
extensions of the same structural logic.

\subsection{Framework as the Meta-Layer of the Entire System}

The Flexion Framework occupies a unique and central position:

\begin{itemize}
    \item \textbf{Above the fundamental theories} — as the coordinating meta-architecture,
    \item \textbf{Below all applied disciplines} — as the essential mathematical foundation,
    \item \textbf{At the core of Flexion Science} — as the unifying structural backbone.
\end{itemize}

The Framework ensures:

\begin{itemize}
    \item structural unity across all Flexion Sciences,
    \item mathematical consistency across all theories,
    \item universal applicability of $\Delta$–$\Phi$–$M$–$\kappa$,
    \item compatibility between geometry, time, fields, and entanglement,
    \item coherent interaction between independent structural systems,
    \item closure of the structural cycle for every form of existence,
    \item consistent boundary conditions for genesis and collapse.
\end{itemize}

FFW is not “a theory about theories” — it is the \textit{operating system of structural 
existence}, the meta-layer that binds the entire Flexion scientific ecosystem into one 
complete, unified model of reality.

\appendix

\section*{Appendix A — Mathematical Notes}
\addcontentsline{toc}{section}{Appendix A — Mathematical Notes}

This appendix provides the core mathematical expressions used throughout 
Flexion Framework V1.2. These formulas define the universal evolution rules,
geometric limits, temporal behavior, field dynamics, and collapse boundaries
shared across all fundamental Flexion theories.

\subsection*{A.1 State Vector}

\[
X = (\Delta,\ \Phi,\ M,\ \kappa)
\]

where each variable obeys the universal evolution law:
\[
\frac{dX}{dt} = F(X).
\]

\subsection*{A.2 Structural Fields}

\[
F(X) = (F_\Delta,\ F_\Phi,\ F_M,\ F_\kappa)
\]

with:
\[
\frac{d\Delta}{dt} = F_\Delta,\quad
\frac{d\Phi}{dt} = F_\Phi,\quad
\frac{dM}{dt} = F_M,\quad
\frac{d\kappa}{dt} = F_\kappa.
\]

\subsection*{A.3 Time Emergence}

\[
t \propto M,
\qquad
\frac{dM}{dt} > 0 \quad \Rightarrow \quad t > 0.
\]

Temporal collapse:
\[
\frac{dM}{dt} \to 0.
\]

\subsection*{A.4 Metric Formation (Space)}

\[
g_{ij} = G(\Delta,\ \Phi,\ \kappa)
\]

Geometric collapse:
\[
K \to \infty \quad (\kappa \to 0).
\]

\subsection*{A.5 Entanglement Conditions}

Curvature compatibility:
\[
|C_1 - C_2| < \epsilon_C
\]

Drift alignment:
\[
\mu_1 \parallel \mu_2
\]

Shared invariant:
\[
I_{12} > 0.
\]

\subsection*{A.6 Entanglement Operator}

\[
(X_1', X_2') = \mathcal{E}(X_1, X_2)
\]

Curvature coupling:
\[
C_1' = C_1 + \alpha(C_2 - C_1)
\]

Memory coupling:
\[
M_i' = M_i + \gamma I_{12}
\]

Stability coupling:
\[
\kappa_i' = \kappa_i + \delta f(I_{12})
\]

\subsection*{A.7 Collapse Boundary}

\[
\kappa = 0
\]

Non-structure:
\[
\kappa < 0,\quad X \to \emptyset.
\]

Collapse divergence:
\[
g_{ij} \to \text{undefined},\qquad
K \to \infty,\qquad
t \to 0.
\]

\subsection*{A.8 Structural Cycle (Closed Loop)}

\[
\boxed{
\text{Genesis → Dynamics → Space → Time → Fields → Entanglement → Collapse → Non-Structure → Genesis}
}
\]

\section*{Appendix B — Notation and Glossary}
\addcontentsline{toc}{section}{Appendix B — Notation and Glossary}

\subsection*{Core Variables}

\begin{itemize}
    \item \textbf{$\Delta$ (Delta)} — Deviation; structural asymmetry; source of form.
    \item \textbf{$\Phi$ (Phi)} — Structural Tension; deformation energy.
    \item \textbf{$M$ (Memory)} — Irreversible structural history; generator of time.
    \item \textbf{$\kappa$ (Kappa)} — Contractivity; viability; structural stability.
\end{itemize}

\subsection*{Derived Quantities}

\begin{itemize}
    \item \textbf{$X$} — State vector $(\Delta,\ \Phi,\ M,\ \kappa)$.
    \item \textbf{$F(X)$} — Flexion Field; set of structural forces.
    \item \textbf{$F_\Delta,\ F_\Phi,\ F_M,\ F_\kappa$} — Field components for each variable.
    \item \textbf{$g_{ij}$} — Structural metric tensor.
    \item \textbf{$K$} — Geometric curvature.
    \item \textbf{$K_{T}$} — Temporal curvature.
    \item \textbf{$C$} — Curvature profile (field-level geometry).
    \item \textbf{$\mu$} — Structural drift vector.
    \item \textbf{$I_{12}$} — Shared invariant in entanglement.
    \item \textbf{ES} — Entanglement Strength.
    \item \textbf{ED} — Entanglement Depth.
\end{itemize}

\subsection*{Domains and Boundaries}

\begin{itemize}
    \item \textbf{Viability Domain:}
    \[
    \kappa > 0
    \]
    \item \textbf{Collapse Boundary:}
    \[
    \kappa = 0
    \]
    \item \textbf{Non-Structure Domain:}
    \[
    \kappa < 0
    \]
\end{itemize}

\subsection*{Structural Objects}

\begin{itemize}
    \item $S_e$ — Entangled Space.
    \item $T_e$ — Entangled Time.
    \item $X_e$ — Emergent Structure.
\end{itemize}

\subsection*{Theories}

\begin{itemize}
    \item FGT — Flexion Genesis
    \item FD — Flexion Dynamics
    \item FST — Flexion Space Theory
    \item FTT — Flexion Time Theory
    \item FFT — Flexion Field Theory
    \item FET — Flexion Entanglement Theory
    \item FCT — Flexion Collapse Theory
\end{itemize}

\subsection*{Cycle}

\[
\text{Genesis → Dynamics → Space → Time → Fields → Entanglement → Collapse → Non-Structure → Genesis}
\]

\end{document}
